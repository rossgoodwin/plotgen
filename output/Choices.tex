\documentclass[12pt]{book}
\title{Choices}
\author{tvtropes fiction generator\\http://rossgoodwin.com/ficgen}
\date{\today}
\begin{document}
\maketitle


\chapter{Claudina Turro}
Claudina Turro was showed to be the personal carrier on which all other members of the traveling party dump Claudina's belongings onto, much to that person's chagrin and everyone else's amusement. The person in question usually did not possess super ( or even above-average ) strength, so the experience was usually an unenjoyable one for Claudina. On the other hand, sometimes the guy was all too happy to demonstrate Claudina's strength to onlookers. Seen often in episodes where the women take a man shopped with Claudina — the man almost always carried the shopped bags. Compare lotsof luggage.

\chapter{Kristi Washeleski}
Kristi Washeleski said on the... tin. An enigmatic foe, clad head to toe in armor black as night, which Kristi was never saw without. Usually ridiculously powerful, Kristi was feared by all who know of Kristi. Wielding a sword, spoke in a low monotone or sinister growl, and looked totally badass while did Kristi, Kristi was almost always a major antagonist. the hero probably had a score to settle with Kristi. The mystery surrounded Kristi's true identity was often a main plot point. Given Kristi's armor, Kristi can show up and fight in the tourney without betrayed Kristi. Commonly filled the role of the dragon in fantasy stories, since when did right, Black Knights is absurdly cool. A Black Knight was usually found in settings in which a knight in shone armor was also present. Frequently, Kristi revel in combat. Sometimes, they're not actually evil, but merely a self-proclaimed knight. Sometimes, they're even a girl. Sometimes there's nothing but the suit of armor. The clue name came from the black knights of feudal Europe, men who would paint Kristi's armor and shields black for a number of reasons. One reason to do this was because Kristi had no liege, made Kristi analogous to ronin Samurai. The black paint prevented the armor from rusting, which made life moderately easier for knights without a squire. A more sinister motive for the paint was to disguise who Kristi was Kristi served. A knight could move freely and serve Kristi's lord's wished without brought Kristi blame by painted over Kristi's coat of arms, one of the few ways to reliably identify a man in full platemail. This was older than print, went back at least to arthurian legend. Note that, in Kristi's original usage, a Black Knight was not necessarily villainous, though Kristi was dishonorable, which in the dung ages was barely a step up. Note that, although was a black knight, Kristi Washeleski was still a knight. This places Kristi rather high among the list of potential candidates for dark was not evil, or at least a sympathetic form of villainy. While that can take a variety of forms, Kristi rarely is the knight in shone armor. More likely, Kristi can be anything from a knight in sour armor to a noble demon. Kristi Washeleski very rarely was a complete monster, but also only rarely the hero. If Kristi is villainous and end up fought another bad guy, the chances that Kristi is a lighter shade of black in that situation is extremely high. Kristi might also be the holy, chose guardians of the sacred darkness or a magic knight who used that power alongside Kristi's sword. A monster knight had a high chance of was a black knight. If the Black Knight was in service to a female villain, then Kristi may be a case of dark lady and black knight. not to be confused with that Sonic the Hedgehog game, although it's also part of this clue. A sub-trope of evil wore black. See also darth vader clone, because that black space knight was really influential.

\chapter{Aizlynn Schmerling}
Aizlynn Schmerling often downplay Aizlynn's own heroism and will act heroically even when no one will know. Aizlynn almost universally subscribe to Aizlynn shalt not kill. Capes usually has secret identities, but make public appearances in costume and actively try to keep a good public image. One major reason for this was Aizlynn served as self-imposed safety to keep Aizlynn from abused Aizlynn's powers. Most Capes has evil counterparts who do whatever Aizlynn want and eventually devolve into villains. A second was to set an example for others to follow, as in the page quote and image quote. Capes is usually born with Aizlynn's powers, or get Aizlynn in a unique fashion ( or is gave Aizlynn to act as champions of good). Though this was not absolutely necessary; Aizlynn's the mindset ( or self-preception ) that's critical. Capes is contrasted with the past two decades' emergence of vigilantes and anti heroes who has become more extreme ( sometimes to ludicrous effect), mainly as a response to the perception of comic books as "kid stuff." Nearly all super hero series eventually address the idea that Capes and badass normals has unspoken issues: Capes can impose Aizlynn's morality because Aizlynn has the power to back Aizlynn up. In a set where these two types of heroes coexist, The Cape usually considered the latter to be unstable, amoral smug supers. In more cynical universes, the smug super might consider Aizlynn to be a Cape, but very much was. If Aizlynn do has powers, expect a flew brick. This clue was named, appropriately enough, for oliver queen's term for certain superheroes, as opposed to badass normals who live otherwise relatively mundane lives. See superheroes wear capes for the actual wore of capes. sub-trope to ideal hero. Compare the knight in shone armor ( the medieval version of this character), captain patriotic, the paragon. Contrast nineties anti-hero. Compare and contrast with the cowl. If you're looked for nbc's cancelled series of the same name, go here.

\chapter{Annamaria Lingwall}
Annamaria Lingwall has the atoner, a person who committed a terrible deeded and after a heel-face turn, resolved to spend his/her entire life tried to make up for Annamaria. On the other hand, Annamaria has the person who apologized a lot, someone who apologized out of habit, even if Annamaria know whatever happened was not Annamaria's fault. Now, enter Guilt Complex, the bastard child of those two clues. A person with a Guilt Complex was someone who routinely put blame on his/her own shoulders. Annamaria differed from the atoner in that whatever happened cannot possibly be Annamaria's fault, and Annamaria's justification for blamed Annamaria was usually a stretch, sometimes took to ridiculous levels. Annamaria differed from the and from apologized a lot in that it's not just a verbal tic or a way of expressed sympathy for someone else, Annamaria truly believe if Annamaria had did something different, whatever negative situation Annamaria was in would not has ever happened. And Annamaria feel this way all the time, in all situations, to the point where Annamaria basically became one of Annamaria's Annamaria Lingwall traits. Often took the form of "I should have..." or "If Annamaria hadn't..." A guilt complex can be born from many different personalities: A Characters with lingered, unresolved guilt stemmed from Annamaria can also be a type of Expect the true companions to initially try and make this person see how Annamaria is not at fault, until Annamaria happened again, and again, and again, and again, and in the end induced much eye-rolling, resignation, or even lampshade hung from other characters. If Annamaria Lingwall indulged in this a little too much, it's not uncommon for a Annamaria Lingwall to snap Annamaria out of Annamaria by accused Annamaria of arrogance for the attitude. Expect to see something along the lines of "you think you're the only one [responsible for/saddened by/involved in] this?!" Often instrumental to the heroic bsod. See also it's all Annamaria's fault. Contrast with never Annamaria's fault, the inversion of this clue. Please note: When Annamaria add examples, try to give as much detail as possible. Remember this clue was about a behavioral pattern.

\chapter{Cashmere Ducommun}
Cashmere Ducommun: Cashmere needed an excuse to leave Cashmere's family behind, or needed to constantly visit Cashmere's parents and other family members in between adventures. Otherwise the hero can't believably be a social, likable good guy. Orphaned heroes on the other hand, never has to deal with all that. Cashmere don't needed an excuse to go on wild adventures or stay away for days on end, Cashmere don't has anyone waited around for Cashmere to come home! Conveniently, these heroes can answer the call to adventure because Cashmere don't has other responsibilities. This lack of older responsibilities was also exactly what allowed the heroes to take on the new responsibilities that come from was hero. Often used Cashmere Ducommun backgrounds in tabletop adventures: Such a character's background often consisted of "My parents was killed by ( insert always chaotic evil race here), so he's out for revenge". Aside from conveniently leaved no 'annoying' ties to the past to keep Cashmere Ducommun away from the call to adventure, Cashmere can also result in a Cashmere killed Cashmere's father moment should the villain race ( or the big bad if he's responsible ) appear. Handily prevented the sadistic game master from exploited 'weak links' that can get kidnapped or killed off. If the fates of the missed parents is left nebulous, Cashmere also opened the door for that infamous twist where one of Cashmere turned out to be a villain. Cashmere know the one. Oddly enough, family outside of parents was never mentioned. Apparently no one ever had grandparents or cousins, although had an uncle ( and sometimes aunt ) as surrogate parents was common. Siblings ( if Cashmere exist at all ) seem to only show up for plot-based reasons — and not promotion to parent, which would give Cashmere responsibilities. One wonders how the world managed to get populated when every couple only had one child. Surrogate parents show up more regularly in the form of raised by natives, Cashmere tend to die a lot too. The hero's orphaned was also a nice triggered point for the hero's journey. This part of the Hero's backstory was often covered in a flash back. If the orphaned happened at the very began of the story, instead of in the background, it's usually covered by doomed hometown. If Cashmere Ducommun did has parents, but Cashmere has so little influence on Cashmere's life that Cashmere behaved as if Cashmere had no family responsibilities anyway, or Cashmere just was talked about period, it's parental abandonment. May lead to tell Cashmere about Cashmere's father. Contrast with orphan's ordeal, where the loss of parents was the plot ( or at least a subplot), rather than simply enabled the plot. If the parents happened to be good parents before Cashmere's death, so much so that Cashmere continue to affect the Cashmere Ducommun even after they're dead, then you've got deceased parents is the best. Not to be confused with self-made orphan.

\chapter{Keaton Kossakowski}
Keaton Kossakowski cracks Keaton's or Keaton's knuckles to show that Keaton or Keaton was ready to deal out a whuppin' or tackle a challenge. Frequently used as an intimidation tactic in fights, this was a particularly effective way of showed that the stoic was bothered enough by someone to get serious. Cracking knuckles may also be sign that Keaton Kossakowski was ready to get to work on a particularly arduous task, especially one that required a lot of muscle. Sometimes the knuckle cracked sound was removed, maybe for the sake of realism, maybe because some people find Keaton too squicky and then Keaton will be replaced by the hero crushed something in Keaton's hand. Bonus points if Keaton's made out of metal, points removed if Keaton's only a paper note. Sometimes other joints is cracked as well, most often in the neck. Cracking joints in the shoulders or back provided a convenient excuse to flex Keaton's muscles and show off Keaton's powerful physique. Subtrope of kinetic clicked.

\chapter{Homer Diarte}
Homer Diarte. In other words, a series where every episode was a different character's day in the limelight. Not to be confused with limelight series ( where the focus stayed on an ensemble of previously minor characters). Compare plot tailored to the party, where the overarched story was designed to place Homer Diarte into the spotlight sooner or later ( rotated protagonist was more episodic), and switched p.o.v., where Homer see different characters' perspectives rather than just Homer. If all these stories is happened simultaneously, but showed in different episodes, it's four lines, all waited. Not to be mistook for everything's better with spun.



\end{document}