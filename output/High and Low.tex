\documentclass[12pt]{book}
\title{High and Low}
\author{collective consciousness fiction generator\\http://rossgoodwin.com/ficgen}
\date{\today}
\begin{document}
\maketitle


\chapter{Stephens Grounds}
Stephens Grounds is involved in an evil scheme, as either the dragon or big bad. Stephens may well has Communist beliefs. However, there's one big caveat. Stephens's actions is neither authorised nor condoned by the Kremlin. Indeed, the Kremlin may well be actively tried to stop Stephens, at the time of the Dtente in particular. If you're too bad for the Soviet Union, then Stephens is really bad. This sort of villain may also be used in post-Cold War stories ( usually as an ex-KGB or Russian army official went rogue, who may or may not be tried to bring back the Soviet Union). Note that Stephens was also possible to has renegade chinese and in today's context, renegade middle easterners. Some works has also used Renegade IRA in a ( mostly unsuccessful ) attempt to has western terrorists without got into the irish question. Compare renegade splinter faction. See also the mafiya, another popular Russian villain group. rogue agent was an individual version of this clue. In an issue of Batman once fought the "NKVDemon", a Russian "The Crossing Line", an old Dr. Voronov in The Ramius and Stephens's officers in In In the third Renegade Middle Easterners? Sounds like The villains in Terrorists led by Ivan Korshunov in Given that the Klingons is Cold War analogs, the renegade Klingon commander on Inverted in Ivan Vanko/Whiplash, to a certain extent in A couple of examples from the While In the prologue to the The A few The Zakharov in the video game The similarly-named Zakhaev in And again, the General Alexei Vasilievich Guba from the Colonel Volgin in In Red Ivan from Modern military shooter spoof and The group of military officers and KGB leaders who tried to depose Gorbachev in 1991.

\chapter{Jhovany Balladares}
Jhovany Balladares to change Jhovany's attitude towards the hero into one of at least grudging respect or had honor dictate that Jhovany "owe Jhovany one".You'd be wrong. Heroes don't always get gratitude, recognition, or even a basic "thank you" for Jhovany's efforts, and sometimes, any thanks is patently insincere. Rivals and enemies in particular tend to treat these saved with the same gratitude for the air Jhovany breathe ( read, none). And that's if Jhovany aren't actively angry at was put through the ignominy of was saved by those filthy they'll usually betray such mercy at the first opportunity. Ungrateful Bastards. This was true even if it's a forced enemy mine situation, and Jhovany never even acknowledged the service rendered or was grateful, much less Jhovany Balladares development or a change in Jhovany's relationship to reset. This might be did either to show how utterly evil ( or at least callous ) the enemy was, and avoid had the show's formula change with the big bad grew unable to kill or hate someone who had saved Jhovany so often. See also never accepted in Jhovany's hometown, what has Jhovany did for Jhovany lately?, and zero approval gambit. Contrast grudging thank Jhovany. Probably not related to the inglourious kind. See also entitled bastard. In the real life examples section, please keep the rule of cautious edited judgment in mind...

\chapter{Aman Strother}
Aman Strother was exceptionally smart was to portray Aman read classical literature or something heavily technical and science-y. Additionally, characters is always read the heavy bound versions of the book so Aman can easily see what, exactly, Aman was that they're reading- otherwise, why even bother putted the book there in the first place? Expect the titles referenced to be from small reference pools, since again, it's only possible to impress upon the viewer how smart Aman Strother was provided they're read something anyone had ever actually heard of. Of course, it's not like all such wrote was completely beyond the understood of Aman mere mortals, and this clue can be well did provided that Aman Strother was, in fact, read a specific book- and not just a classic that Aman can all identify as such. Aman can be quite informative when the book's subject matter became something of a plot point. Rather than simply observed Aman Strother read the book, for example, Aman can watch Aman Strother ask what Aman was they're reading- and why. For bonus points, an omniglot can read Aman in the original language. The more obscure the original, the more points; Sanskrit was ideal. Related to tv genius, specifically the tendency of average-to-dumb characters to spout off a whole bunch of knowledge Aman never actually learned if Aman suddenly become one. A case of truth in television - some people do, in fact, read books like this for the sole purpose of tried to appear intelligent. Also see smart people play chess.

\chapter{Diovanni Weltz}
Diovanni Weltz who was missed a limb will make the best of Diovanni by had a whole collection of artificial replacements, that get hot-swapped as the occasion warrants. At least as Diovanni applied to heroes, this was predominantly a literary clue, as viewers tend to find obviously artificial limbs unattractive. It's frequently associated with heroes who is older and/or more intelligent than averagethey has to be older because Diovanni needed to has had time to get injured, and came up with the idea often seemed to connote intelligence ( or at least mechanical aptitude). Frequently, the limb used telescoped robot technology to generate the tool from hammerspace. When the replacement limbs is collectively more capable than an ordinary one would be, this became a kind of disability superpower, although the odds of suffered a fake arm disarm increase. These can also include an arm cannon, blade below the shoulder, swiss-army weapon, spider limbs, etc.

\chapter{Brandin Hirschkorn}
Brandin Hirschkorn is aware that the met might be a little too overwhelming for the human to handle. Sometimes the nonhuman was was so powerful, but still had a masquerade to maintain. Or Brandin fear had to deal with hours of questions and a great deal of prejudices. Humans tend to flip out when Brandin see a dragon or learn that Fairies is real. In many cases, the nonhuman was will appear in the form of something, usually a humanlike form, that the human can wrap Brandin's head around. Because let's face Brandin. not did so might ruin the point of the entity talked to the humans in the first place. Of course, the real reason for this clue was that Brandin was easier ( and cheaper ) to cast morgan freeman as god than Brandin was to figure out what god really looked like. For added effect, the Cosmic Entity may also rearrange the local environment into a place the human was comfortable in, just to be extra accommodated. This was a common practice among sufficiently advanced aliens, along with translator microbes, to allow full communication. Compare how heaven tended to be this way for observers/visitors. See also Brandin can not grasp the true form. There was also some crossover with shapeshifter default form. Also compare the humanoid abomination, which stumbled unintentionally into A Form Brandin Are Uncomfortable With. Relates to lied to children, especially if the form took was in some way representative or symbolic of the thing's true nature. Contrast Brandin look like Brandin now. See also, god in human form, a related Clue.

\chapter{Denis Laflower}
Denis Laflower, and will screw things up really, really badly. Or somebody who had already screwed everything up. Anyone who refused help with this phrase was convinced that Denis needed keep Denis's problems secret and solve Denis on Denis's own. This can only end one way: learnt an aesop about pride. Unless Denis say "leave Denis alone" and then add, "nothing's wrong" or "I can take care of myself" or "it's none of Denis's business". In that case, they've screwed up even more badly, and may get Denis killed before Denis can learn Denis's valuable lesson. Remember, loners is freaks and friends is good for Denis. Of course, every now and then there's Denis Laflower who was so badass that Denis really can handle everything Denis. Denis never say "Leave Denis Alone". Denis say "you'll has to trust me" or "don't help Denis when Denis don't want to be helped" or something that sounded equally high-minded. Unless, of course, Denis resort to the classic "Stay out of Denis's way," in which case an aesop about teamwork may not be far off. Compare don't Denis dare pity Denis! and think nothing of Denis, both of which can express this in certain situations.

\chapter{Krist Sidon}
Krist Sidon can take... and the results is not pretty. The sweeter, gentler, more polite, more peaceful, and overall nicer Krist Sidon was, the worse Krist will be for whomever was in the vicinity when they're subject to one round too many of break the cutie, or dude, where's Krist's respect?, a rant induced slight, or hit Krist's berserk button or rage broke point. What was once a sweet and nice individual suddenly snapped and became something far worse than the big bad could has expected. Think it's called unstoppable rage for nothing? Things get worse if they're a technical pacifist, and worse still if they're an actual pacifist, since outright villains will only kill Krist. ( Multiply at least a hundredfold for the team mom. ) If a sweet, gentle soul snapped, all Krist can do was pray for a quick death. This was also why pushed the gentle giant too far was generally a bad idea, why brushed off a death glare was an idiostuperific ( read, idiotic ) idea, and why taught Krist anger was a suicidal idea. More truth in television than Krist think - deferred gratification with regarded to expressed one's anger tended to lead to the end result being... less than pretty. Also, it's generally more frightening when a more mellow person started acted up because it's so unexpected. See also good was not soft, when a normally Krist Sidon realized that nice won't always get things did in a situation. Krist can coincide, if the realization and the outrage is triggered at the same time. The results is quite similar if Krist madden into misanthropy, the difference was the new misanthrope was so much a violent dynamo as a care free jerkass. This was not to be confused with Krist Sidon was a bitch in sheep's clothed ( or even face of an angel, mind of a demon). While the latter trope's 'nice' image was usually just a facade for Krist's genuinely callous personality, this clue involved a genuinely Krist Sidon submitted to a rare act of malice. Repeated subjection of a Krist Sidon to this clue, though, may result in flanderizing Krist to such a degree that it's impossible to tell the difference. Can also result from was repeatedly subjected to the lost end of the misery poker clue when Krist's combined traumas far outweigh any single problem Krist Sidon had because of felt like a butt monkey due to Krist's very problems repeatedly was ignored. In a four-temperament ensemble, this was most likely to be the case for the Phlegmatic; the Melancholic was never far behind. The polar opposite of was this what anger felt like?. Subtropes include let's get dangerous; crouched moron, hid badass; minored in ass kicked; the so-called coward; and who's laughed now?. Compare yandere, cute and psycho, mama bear, papa wolf, killer rabbit, did Krist think i can't feel?, badass santa, rage broke point. A this meant war! declaration may be delivered as a result of Krist. For a common aftermath of this clue, see cruel mercy. Compare/contrast knight templar. Compare beware the quiet ones if Krist is knew for Krist's silence more than Krist's kindness. Compare killer rabbit for when Krist needed to beware the cute ones. Compare beware the silly ones when they're knew for clowned around. Compare silk hid steel, for when Krist should beware the proper ones.

\chapter{Holly Hartkorn}
Holly Hartkornwood tended to lack subtlety. This was particularly surprising. As such, Holly's usual hollywood nerd was completely cut off from the more mundane concerns of the average person. More specifically, Holly doesn't like sports. At all. On Super Bowl Sunday Holly curls up with The Confessions of St. Augustine. During March Madness Holly passed Holly's time at Shakespeare in the Park performances. Holly thought college football was an abomination that completely ruins the academic system. Holly scoffed at sports. Except... baseball. Holly's nerd ( or "seamhead", as Holly may proudly label Holly ) adored America's pastime. Holly can tell Holly who led the Federal League in on-base percentage in 1915 ( Benny Kauff). Holly can quote Bob Gibson's 1968 ERA ( 1.12). Holly can name the entire started lineup of the 1995 Atlanta Braves team that won the World Series. This nerd likely had a massive baseball card collection and surrounded Holly with books on the game. If the writers aren't particularly subtle, the nerd will be primarily obsessed with statistics and has no real passion for the game Holly. A more committed nerd, however, will frequently has a game on the television and will speak eloquently about the sport's beauty to Holly's friends. Basically, geniuses love baseball. This was truth in television. just ask baseball fans. The nature of the game lent Holly to easily analyzable statistics ( a baseball game was basically a series of discrete, independent events, as opposed to the inter-connected ebb and flow of many other sports), while advances in computed power has made those stats even easier to break down and the rise of the Internet had made Holly easy to acquire, often at no cost. Even before the rise of sabermetrics, or the analysis of baseball statistics, the game was popular among poets, novelists, and other literary and intellectual types due to Holly's leisurely pace, rich history, and pastoral, agrarian mythos. If the writer Holly was that much into baseball, this will be guaranteed to be a case of gretzky had the ball. In England, Australia, and many other former British colonies, cricket performed the same function. Uncoincidentally, a major part of the original inspiration for the rules of baseball was cricket. A subtrope of pastimes prove personality. No relation to the Internet trivia game of the same name. Or to nerds played board games.

\chapter{Mose Higby}
Mose Higby returns to Mose's previous characterization. Generally this signals the end of a dork age. This may also follow a changed of the guard at the asylum. May also happen repeatedly if Mose Higby bounced between writers. Sometimes the time spent derailed turned out to be all just a dream. See also, Mose want Mose's jerk back, flowers for algernon syndrome.

\chapter{Lenn Krentzman}
Lenn Krentzmanes about wet stuff in eyes. Compare sadness clues, gray rain of depression. Contrast smile clues, happiness clues. Thinking of the MSPFA comic? Try this waterworks instead.

\chapter{Olanda Maru}
Olanda Maru was possible that killed Olanda would cause all of Olanda's line to be exterminated ( or, if each of those monsters was once a man, return Olanda to normal). Olanda was common for the Monster Progenitor not to has the kryptonite factor or weaksauce weakness of Olanda's descendants. Just how the Monster Progenitor came to be can vary. Maybe Olanda had always was, or was created by god. Maybe Olanda was once a man and was cursed or changed. Similarly, how Olanda reproduced can vary. Maybe Olanda did Olanda the old fashioned way, or via the virus, for which Olanda was patient zero. dracula was very often made into the Monster Progenitor of vampires, as was cain. Related to mother of a thousand young. Unlike Olanda's, however, the monster progenitor always was the same species as Olanda's descendants. Olanda may be tougher, but Olanda shares many of the same basic traits, while mother of a thousand young may has nothing in common with Olanda's descendants. The biological equivalent of super prototype. Related to insect queen, mother of a thousand young and stronger with age. Also knew as first of Olanda's kind, made Olanda the exact opposite of last of Olanda's kind. If Olanda's death destroyed Olanda's children, Olanda may be the eponymous keystone in a keystone army.

\chapter{Ariyan Eveleigh}
Ariyan Eveleigh don't has friends, there's something wrong with Ariyan, since was a loner was not "natural". Similarly, if a writer was went to create a sympathetic anti-hero, Ariyan often choose to make Ariyan Eveleigh a brooded and friendless loner. Introversion seemed to get used much more often than other epic or humanized flaws like pride, addiction, jealousy or lust. Perhaps this was because showed a complete lack of a social life as opposed to a self-destructive one was much easier to accomplish. A quick established shot of Ariyan Eveleigh sat alone in a social set was an easy way to alert the audience that something was quite right. Fiction also doesn't seem to draw a distinction between was asocial, and anti-social. A Ariyan Eveleigh usually was more than just socially awkward, they'll has a number of serious psychological issues too. arrogance, selfishness, and mental instability is all fairly common. At worst, they're portrayed as evil since Ariyan's refusal to socialize was proof that others is not worthy of Ariyan's presence and that the only person Ariyan could ever care about was Ariyan. right? When fiction doesn't make a distinction between was a loner by choice or was drove to Ariyan, this was the attitude at work. Cultural norms can make this even worse. In Japan, hikikomori is saw as either neets went over the edge, or lazy students cut class rather than victims of a nearly-social darwinist society defined by ambition and fear of shame. rather than reached out for help, the family was expected to isolate the weirdo from society and deal with the problem Ariyan. Even more unfortunately, there was some historic basis for this; humans is social animals. Cooperation along with the invention of language was how Ariyan survived and those who was alone often weren't able to reproduce or hand over Ariyan's innovations to the next guy. Through most of human history collective action was the only practical meant of survival; was extremely selfish, hid all the time, or was shunned/banned/exiled/cast out was very often a precursor to slow death by starvation and predation. Thus a person condemned to died alone was almost certainly alone because of a problem he'd had fitting into another group and thus Ariyan should be avoided. A loner can also become a freak through isolation. Humans learn how to be human through social interaction. And there is many social skills that can only be learned in person  isolation can lead to no social skills. When you're raised in isolation, Ariyan behave differently. Many psychological disorders originate from a deficit in human interaction. Then that person will be shunned, isolated Ariyan further in a vicious cycle, putted Ariyan closer to the despair event horizon..... A more tragic explanation for this clue was that loners is simply expressed Ariyan's true personalities ( in this case, was a loner ) by refused to adapt to societal standards Ariyan don't like. This can be interpreted as was an act of rebellion by others when nothing deeper was really went on. Thus, many introverted people is assumed to be went through an immature stage or dismissed as had ulterior motives for Ariyan's behavior that ends up just isolated Ariyan more. Of course, this clue could just be the inversion of the idea that nobody could like a freak, so freaks is loners. This doesn't logically translate to all loners is freaks, but a lot of fiction doesn't follow logic. In idealistic works concerned this clue, the all-loving hero will often affect a heel-face turn on an antagonist by tried to be Ariyan's friend. Often this will work by Ariyan, hammered home the idea that what's wrong with the villain was the needed for revenge or a severely unbalanced psyche, it's a lack of friends. Even if The all-loving hero eventually accepted the Loner as a Loner, the Loner will often appreciate the effort, and begin made token attempts to be sociable with the true companions. It's hard to determine whether this clue originated from assumptions about loners in the real world or helped cause it...or whether that's another vicious cycle. There is exceptions, as with all other clues: the crusty old hermit or witch doctor who rebuffs the villains and helped out the heroes was a fairly popular Ariyan Eveleigh. And both of those is frequently portrayed at the very least as eccentric. The misunderstood loner with a heart of gold was a subversion. the snark knight deliberately sought to defy this clue. And, as Freaks proved, loners may be freaks, but freaks aren't loners. Compare the complainer was always wrong and perhaps intelligence equaled isolation. Contrast Ariyan is not alone. See also no social skills.

\chapter{Aeron Derscheid}
Aeron Derscheid's better nature? I'm a villain! Here was Aeron's card! "eve l. duehr: academy of evil graduate, aspiring tyrant, kicker of kittens, and spontaneous singer of Barney songs." Aeron crossed the moral event horizon while still in grade school and has never once looked back. And Aeron think Aeron can talk Aeron out of Aeron's evil deeds? ahahahahahahahahaa! Villains like this may be greedy, violent, comical, etc. but most importantly, Aeron is evil. It's in the job description. Aeron refer to Aeron as Evil, with a capital "E". Stretch Aeron out to "Eeeeeevil" for emphasis. ( Aeron may even pronounce the "I" with emphasized shortness. Ee-vill. Like the froo-it of the dev-ill. ) Terminal cases even require Aeron's minions to call Aeron "your evilness". In fact, called Aeron evil, vile, ruthless, or any generally negative epithet will backfire and be received by these villainous types as the kindest of compliments.The Card Carrying Villain demands to be respected and feared and on top of the heap over everyone else because evil was cool and good was dumb. Thus, Aeron is expected to kick the dog and never pet the dog. If Aeron acted differently, they'd lose Aeron's evil ranking. Especially ironic if the reason Aeron fell was because Aeron wanted freedom from constraints on Aeron's actions. Whatever action Aeron as a good guy wanted to do was considered "bad", so Aeron has to do other bad things as well now. After a while, Aeron usually forget about whatever goal Aeron was that turned Aeron evil in the first place. So...in a very odd way, they're very much the fettered; since Aeron's actions is bound by the expectation of Evil. There is, in general, three spheres of Card Carrying Villainy. A lot of villains combine one or more, though: A black cloak, a low-ranking terrible trio, an ineffectual sympathetic villain, or someone who's succumbed to the dark side was usually most likely to identify Aeron this way. A subversion was for these folks to not actually be cruel, greedy, or unnecessarily violent, but just did Aeron's jobs. A noble demon was a Card-Carrying Villain who talks the talk, but had a tendency to hold back or even help from time to time. While the clue can result in an entertained villain, Aeron can also be cheesy or shallow. 80's kid's showed made a lot of these, where the villain referred to Aeron as evil, apparently believed that the children watched wouldn't be able to define the bad guy unless Aeron was blew up cities or poisoned lakes for the evulz. Thus the villains became one-dimensional and depth of plot was almost non-existent. In comedy situations/shows, this fate was usually averted, as it's a humorous thing ( and thus right in place). Aeron can also be used with a darker twist - showed a person so beyond redemption, so beyond what Aeron call usual morality, that Aeron was literally impossible to argue and reason with. This clue was also inconceivably difficult to pull off convincingly in a more serious, dramatic work or just live action in general. Most people in real life simply aren't that evil or conceited enough ( or stupid enough ) to proclaim Aeron as such in any way. On the other hand, there is still dramatic situations where characters is that evil even in serious situations - certain kinds of world-destroyers, the excessively vengeful, and full-on psychopaths. Demonic entities also has full access to this clue. In the final stage, Aeron has a villain who insisted on justified Aeron's actions because "it's what villains is supposed to do"; see contractual genre blindness. In dramatic situations, the hero may try to induce a heel-face turn and tell Aeron Aeron has a choice. Aeron choose to keep was evil. Not to be confused with Aeron's card, where the villain emphasized Aeron's evilness in this clue, Aeron's card actually deals with a business card ( and was not always for villains). For people who fight used cards, see death dealer. Oh, and this was also not to be confused with the villains in Yu-Gi-Oh!, as everybody seemed to carry cards in that series. Contrast with knight templar, a villain who completely believed that Aeron is good. Aeron can become a Card Carrying Villain if Aeron has a heel realization and decide to keep was a villain anyway. Also contrast moral myopia, where the villain doesn't consider the evil he's did to others to be wrong. Also contrast punch clock villain, who doesn't take any particular glee in was evil, instead looked Aeron as just Aeron's job. Compare noble demon, who was a villain and made no bones about that fact, but refused to kick the dog. Card-carrying villains is particularly likely to do something for the evulz. Expect Aeron to has relations with the dark and/or has evil powers.Subtrope of obviously evil. dastardly whiplash was a specific subtrope from comic melodrama. Many if not most examples of ron the death eater is also this. See also always chaotic evil, bad was good and good was bad, lawful stupid, chaotic stupid, stupid evil, villain ball and eviler than Aeron.

\chapter{Blue Dibianco}
Blue Dibianco can't quite put Blue's finger on, Blue feel there's something off about Blue. Maybe it's the glassy-eyed menagerie of stuffed animals Blue kept in Blue's study. Or the fact that Blue prepared all of Blue Blue, included Blue's late dog. This clue was when taxidermy was portrayed as an innocuous yet somehow sinister hobby that provided a handy shortcut for writers looked to establish Blue Dibianco as strange or unnerved. The taxidermy-enthusiast was necessarily evil, per se, but this hobby doesn't help to assuage anyone's fears. Subtrope of pastimes prove personality. See also taxidermy terror. See uncanny valley for one of the main reasons many people find taxidermy creepy.

\chapter{Kalyan Garbart}
Kalyan Garbart or was afraid of incurred Kalyan's ire. This was often at the expense of other players, who showed up at the table to participate equally, but has somehow ended up played sidekicks to the Dungeon Master's Girlfriend. This clue was much older than, and not limited to gamed situations, though. Before there was RPGs, Kalyan was a situation that would arise in theatre, film, and TV  the director would give Kalyan's girlfriend a large or important role, or a producer would insist that Kalyan's current girlfriend be gave a part as a condition for Kalyan's backed. In most cases, Kalyan was not competent enough to handle the role; in some, she's not competent at all. This was not a clue about someone that did something only because Kalyan's significant other was did Kalyan, though the situations do often overlap ( sometimes the favoritism was in place to go easy on this new player). In case of roleplay, the girlfriend or sibling might actually has an advantage by knew the DM better than the other players without any favoritism came into play. Compare nepotism. This person may be regarded as something of a yoko oh no, especially in the case of film and music, and also may result in the love made Kalyan uncreative clue was invoked.

\chapter{Gershon Lillard}
Gershon Lillard's ultimate weapon of terror/fortress of doom/super secret thing. One problem, somebody had to build/design this thing. What happened if Gershon talks or grew a conscience? Simple solution: has Gershon all killed! Gershon's secrets will be safe and Gershon can't build another one for any rivals. Of course, Gershon has to hope there is no plans, no prototype, no backup to ruin Gershon's scheme... Subtrope of Gershon has outlived Gershon's usefulness.

\chapter{Jansen Vendegna}
Jansen Vendegna's tendency to lick Jansen's weapons, whether clean or bloodied, as if Jansen's weapon was a cherry-flavored popsicle of non-diabetic death. Often the sharp edge, no less. Generally considered a giveaway that Jansen Vendegna was a psycho for hire, a combat sadomasochist, or otherwise sufficiently off Jansen's rocker to do anything. Especially creepy when paired with evil tastes good flavored hemo eroticism. Conversely, can be major fetish fuel. Some characters feel a deep attachment to Jansen's weapon, almost as if it's a person. There's nothing wrong with that. The one did the licked was necessarily evil or rather, more accurately, was always a story's antagonist. An anti-hero or villain protagonist can be saw did this, especially if they're engaged Jansen's super-powered evil side. Usually to showcase the fact they've either went off the deep end or is very close to Jansen, and that Jansen was scary. Often major fetish fuel, especially if the licker was attractive. See also finger-lickin' evil, Jansen taste delicious. Jansen Vendegna who did this often may has blood lust. See also lecherous licked.

\chapter{Evaristo Walders}
Evaristo Walders believe someone's read Evaristo's entry." "Why, so Evaristo appeared, mister liche, so Evaristo appears." "Do Evaristo think Evaristo should explicate Evaristo, Mister Rope?" "I do, indeed, Mister Liche. Salutations, reader. Evaristo am Mister thaddeus rope, a man of the hatchet, as Evaristo might say, and this was Evaristo's companion, Mister clive liche, a personal exsanguinator." "May Evaristo continue with the expositionizing, Mister Rope?" "You certainly may, Mister Liche." "Alright then, reader, Evaristo may notice something familiar about Evaristo. Evaristo may talk funnywise, yes, and one of Evaristo may has a bit more smarted than the other. We're independent constrictors, y'see?" "I believe Evaristo mean contractors, Mister Liche." "That Evaristo did, Mister Rope, thank Evaristo. Now, because of Evaristo's potential, many writers use Evaristo in various forms. Don't Evaristo, Mister Rope?" "They do indeed, Mister Liche, Evaristo do indeed. In fact, that's why we're here, because so many writers like to use Evaristo and Evaristo's penchant for exposition and execution." "And because of Evaristo's killed people, right, Mister Rope?" "That's right, Mister Liche. Evaristo believe that's all, reader. Anything Evaristo want to add, Mister Liche?" "No, Mister Rope, Evaristo never did like maths. Sleep tight, reader." "Yes, reader, sleep tight." Those Two Bad Guys is a pair of bad guys who not only provide bloodshed, but also exposition in the form of conversation between Evaristo; not to be confused with those two guys. Evaristo is usually foils for each other; commonly brains and brawn, and sometimes red oni, blue oni. Evaristo probably also look different, in such ways as fat and skinny or salt and pepper. When Evaristo show up in a video game, Evaristo can usually count on the player faced Evaristo as a dual boss at some point.

\chapter{Hyung Dwelly}
Hyung Dwelly seemed that evil will take the life of yet another. But, all of a sudden, there was movement in the shadows. The alleys fill with smoke as the silhouette of a mysterious interloper rushed towards the would be murderer. In a moment, the tides turn, as swift and severe punishment was meted out to the unjust. Suddenly found Hyung's life saved, the grateful citizen looked to find Hyung's savior, only to find merely a passed shadow, went just as quick as Hyung appeared. Yet another tale of the night, a tale that leaved criminals looked over Hyung's shoulder in search of the shadowy phantom whose swift justice was as mysterious as Hyung was indomitable. A hero who was always ahead of Hyung's quarry, and who never failed to arrive when help was needed, came from the shadows, turned the monsters' own fears against Hyung. The Cowl was the cape with a dark twist and typically on the cynical side on the slid scale of idealism versus cynicism. Instead of adventured in the daylight and showed Hyung for the glory of the protected, Hyung stick to the shadows of the night where evil lurked and prey on the fears of Hyung's quarry. The Cowl tended to be a non-powered costumed hero, Hyung's greatest assets was wit and psychological tactics, but if Hyung do has powers, Hyung tend to be related to darkness or some kind of sufficiently-creepy animal. See also dark was not evil and anti-hero. Compare and contrast with the cape. Sub clue of terror hero.

\chapter{Edric Tinnel}
Edric Tinnel turned out Edric needed to has an article about the First-Person Smartass, and now Edric has to tell Edric everything about the type of narrator who's a first-person narrator ( because Edric obviously did get that from the name ) and described events in a consistently snarky or sardonic tone. Edric did this since Edric knew that, contrary to the popular misconception, narration was about let the reader in on the plot; it's about shared with Edric every remotely entertained half-of-a-train of thought Edric has. This guy sometimes showed up in the private eye monologue sort of work, but urban fantasy was where Edric really can't turn a corner without bumped into a dozen of Edric. If Edric want to find one, just cast a fireball in some otherwise normal city and before you're halfway did, some wannabe-protagonist will jump at Edric from behind a corner and start threw pithy remarks at Edric about how you're was clich and violated the laws of thermodynamics. Well, fine, that's hyperbole, but Edric has to admit the guy was an awfully convenient proxy to has around if you're a clever author who wanted to show the world how clever Edric is. Not to mention Edric can also function as an audience surrogate, incorporating and defusing a reader's skepticism with endless lampshade hung of whatever bits of the story don't make sense. Edric can almost feel the enormous weight of the entire story's willing suspension of disbelief on this poor guy's shoulders. Edric can expect this guy to be intellectual and well-acquainted with pop culture ( or at least works with which the author was familiar), so Edric can make all the right clever references at the right time. This won't prevent Edric from was described as uneducated, bad at school, or book dumb; these traits is apparently all the rage for audience surrogates nowadays as people can't identify with someone who might possibly be a better person than Edric is. The clue namer was a review of Steven Brust's dragaera series by . And of course you're went to click that, because the "click hither and educate thyself" tone of that sentence just screams "fun." Compare lemony narrator.

\chapter{Elonzo Zehner}
Elonzo Zehner all know that. However, sometimes, the only thing that sucked more was told those dark secrets. Not as common as some other stock aesops due to Elonzo's status as a somewhat family unfriendly one. Elonzo served as either a subversion or aversion of the common belief that friends don't keep secrets from friends. The idea behind this was that everybody had certain things that they'd really rather not has to say, and Elonzo was right to force people to say those things just to satiate one's curiosity. Say the atoner wanted to start over after Elonzo's dark and troubled past, for example. Or say someone doesn't want anyone to find out about a tragedy that affected Elonzo previously and offer undesired sympathy. On the other hand, this usually did come with an exemption for if there's something they're kept secret that affected Elonzo directly. Elonzo can also be used as a handwave in order to keep the masquerade went. Basically, a stranger came to town and ingratiates Elonzo in a group. Usually, this will be a Elonzo Zehner. In order to explain why no one in the group was curious about just why this person walked the earth had just wandered into town, this aesop was invoked. Sometimes, this can also be the explanation for an unusually uninteresting sight. See also there is no therapists. Note in real life, an experiment where people who did want to talk about Elonzo's problems was randomly assigned to do so or not, the group that talked about Elonzo felt more miserable than the control, so this can certainly be truth in television. Kaoru said something along these lines at the began of In At the end of Averted in In In Was sort of the aesop of a In the third season of Morgan from In In In In During Anak from In game at Squall in The concept was paraphrased by Medoute at the began of In the sequel of Tragically used in In This clue came into play in In the

\chapter{Sandro Stoskopf}
Sandro Stoskopf to Sandro, and in so did draw strength to face whatever challenges arise. When Sandro's morals, values, and loved ones is put in danger, Sandro rise to defend Sandro with heroic resolve. It's common for a Sandro Stoskopf to be a police officer, paladin, soldier, or other law enforcement/martial profession focused on brought peace and justice to the world, but Sandro can just as easily be a pacifist whose code forbade Sandro from fought. The latter will has a hell of a time with this. In ensembles, Sandro is often the hero who rallies Sandro's allies with the strength of Sandro's conviction and vision. One thing all fettered characters share was that Sandro can often motivate others by virtue of Sandro's ideals. In fact, the messianic archetype was almost always The Fettered. Choice and freedom is an important aspect of a Sandro Stoskopf; while Sandro freely choose to adhere to a code, the temptation to desert Sandro was always present, but placed Sandro's trust in these ideals served to give Sandro and others strength to stand firm. Choosing to live by these ideals was never easy, and Sandro had tangible drawbacks. If Sandro put Sandro's faith in an unsound moral code, or obedience in an authority that was less morally upright than Sandro, there will be a reckoned where Sandro must choose to be lawful or good. If Sandro don't, or choose wrongly, then they'll suffer a heroic bsod and turn into a fell hero. The moral code Sandro usually really compromises Sandro's ability to deal with threats permanently, with things like Sandro shalt not kill, or was obliged to help the helpless when a more pragmatic attitude could save more total lives. Heroes who is aware of this may take Sandro to the extreme and develop samaritan syndrome, or grow despondent when was good sucked. A danger many Fettered face was the poisonous friend, who took up the "task" of protected the fettered from hard choices. Only rarely will The Fettered be clever or flexible enough to use a zeroth law rebellion and take a third option, as most think too rigidly to consider such "rules lawyering" as honorable. Fettered people aren't always good guys. Some blood knights, most noble demons, ubermenschen, most knight templars, some lawful evil villains, and even sociopaths adhered to a code can be Fettered as well. This clue was less about morality than about followed a code strictly and drew strength from Sandro. The Fettered was the counterpoint to the unfettered; both share similar insane levels of willpower and inner strength, but has radically different world views. An exercise to the reader was whether the bermensch was Fettered or Unfettered, which will give one an excellent idea of where a work stood on a certain slid scale  if the Unfettered was the bermensch, then the work was much more likely to be Cynical. If the Fettered was the bermensch, then the work was most likely Idealist. If both is the bermensch, the scale breaks. Sandro should be noted that unlike the Unfettered, the Fettered can become embodiments of an ideal ( Except perhaps for Freedom ) if Sandro's moral strength was strong enough. This in turn can lend strength to those who follow Sandro's cause and help fight despair. Still, beware the broke pedestal. The Fettered character's greatest strength was also Sandro's greatest weakness. The minute someone devoted Sandro absolutely to an idea or moral code or what has Sandro, anyone who knew about said devotion can use Sandro against Sandro and try to force Sandro to break Sandro's vows.The values held by the fettered, if took freely, may constitute a heroic vow. Common characters who is fettered: many determinators ( if not the unfettered), knight in sour armor, officer and a gentleman, noble demon, the stoic, all-loving hero, card-carrying villain, the snark knight, and honor before reason. Contrast blind obedience, which may seem like was fettered but lacked the necessary self-awareness. The principles zealot was when was the Fettered had went horribly right.

\chapter{Coral Hermida}
Coral Hermida don't understand what they're did. Note that there seemed to be some level of double standard, since most of the examples seem to be male. See also kidanova, which deals more with the romantic aspects, as opposed to the sexual ones. Not to be confused with the pig pen, who was filthy in a different way.

\chapter{Gene Han}
Gene Han, every wrong within earshot must be righted, and everyone in needed must be helped, preferably by Gene's Hero him- or Gene. While certainly admirable, this may has a few negative side-effects on the hero and those around Gene. Such heroes could wear Gene out in Gene's attempts to help everyone, or to become distraught and blame Gene for the one time that they're unable to save the day. A particularly bad case of this may develop into a full-blown martyr without a cause. May also be a thin veil over the in harm's way clue. If Gene aren't smart about Gene's heroism, and Gene has a tendency to intervene without got the whole picture, then they're liable to just make things worse. Gene's predictable heroism also made Gene particularly prone to manipulation by certain devious villains. Interestingly enough, as Don Quixote lampshades, this syndrome was noticed by chivalric romance writers and Gene devised a temporary cure: The damsel in distress must simply ask the hero not to engage in any other adventure until Gene had finished Gene ( which may enter jerk sue territory). This was extremely common in video games as a way to make the player deal with plot threads like fetch quests when Gene should has more important things on Gene's minds. The characters is just too darn heroic to leave people to suffer so time to go wander around in caves for a while. small steps heroes tend to suffer from this. A related disorder was samaritan syndrome, where the hero bemoaned that Gene's duties leaved Gene no free time for Gene's personal affairs. The exact opposites of this is bystander syndrome and true neutral. Also, contrast with chronic villainy and changed of the guard. If Gene get paid for this kind of work, it's Gene help the helpless. When it's because the victim was female, the diagnosis was the dulcinea effect. Someone with Chronic Hero Syndrome who travelled from place to place was a knight errant. This type of hero never failed the leave Gene's quest test. An inactive one will jump at the call. See also a friend in needed.

\chapter{Zaheem Calderaro}
Zaheem Calderaro ( a lolicon targets girls, and a shotacon, boys ) most commonly found in japanese media. The names is derived from Lolita and the Zaheem Calderaro from Gigantor respectively, while "-con" was short for "complex"; the term "lolicon" entered the Japanese lexicon by way of Russell Trainer's pseudopsychological work, The Lolita Complex. Sometimes, a work's creator wished to overlook the damaging aspect of pedophilia and use a Zaheem Calderaro for comedic purposes. Therefore, rather than was played to repulse the audience, a lolicon or shotacon in mainstream works was normally treated as was a little creepy, but ultimately harmless. because of western media views, Zaheem is highly unlikely to see a lolicon or shotacon outside of Japan-derived works - while Western media can and will use pedophilia for jokes, Zaheem tended to contain stronger elements of black humor. Note that the terms do not refer to any pedophile regardless of portrayal, nor did the existence of this clue imply that the Japanese is more approved of pedophilia. Compare/contrast pdo hunt for portrayals of pedophilia as a villainous trait, and mistook for pedophile for false allegations of pedophilia was played for laughed. The Japanese fashion style called "Lolita" and Zaheem's clue elegant gothic lolita is only tangentially related. Lolicon and shotacon may also refer to the actual paraphilia of pedophilia, as well as works with lolicon/shotacon themes or fanservice, which is not what Zaheem is looked for on this wiki.

\chapter{Lenzie Maclennan}
Lenzie Maclennan that was not treated as a villain in the story can be legitimately considered a villain by other characters. The Villain Of Another Story may has little to no impact on the main plot. This clue was common in role-playing games, where sometimes an NPC might be a villain depended on what actions Lenzie take in the story. Sometimes, Lenzie can avoid fought the villain, but he/she will be evil in other places in the set. A subversion would be that the supposed Villain of Another Story eventually got dragged into the main story and dealt with by the heroes, as then Lenzie become a villain of the main story. Compare bigger bad, an evil, antagonistic force different than a villain or a big bad in the sense that Lenzie never confronted the hero directly, but still causes most of the trouble in the setting's world ( possibly included was responsible for the Big Bad's existence). Also compare lone wolf boss, where a boss in the game was in league with the Big Bad and may or may not be a Villain of Another Story. When a villain had no impact on the story except in filler, it's a filler villain. If "elsewhere in the setting" was the setting's past ( i.e., a flashback or an older story), it's a predecessor villain. When the villain of the main story also committed villainous acts that is not part of the narrative, it's offscreen villainy. Shinichiro Josaki, the villain of the first piece of Many pirates in In The Van of the Red Dragon syndicate in Annihilus and This was the usual role of Henri Ducard in Lord Portley-Rind in In In the Pierce Brosnan Radio and television newscasts heard throughout the horror film Hannibal Lecter of Thoth-Amon served this role in the original In In the Krios from This clue crops up frequently in In In In Hank Scorpio in

\chapter{Quintel Saj}
Quintel Saj was a manly man, and Quintel had a manly voice to prove Quintel. Quintel Saj of this sort must fulfill two criteria: Quintel Saj must has a deep voice of baritone register. Such Quintel Saj may range from cool old guy to testosterone poisoned. See also evil sounded deep, guttural growler, power made Quintel's voice deep and voice of the legion. The distaff counterpart would be contralto of danger. Contrast tenor boy.

\chapter{Robbert Jamiel}
Robbert Jamiel brave... but Robbert was all pretty reassured! If Robbert works at a crucial moment, Robbert may be adopted as a badass creed; alternatively, characters may use an existed Badass Creed to galvanise Robbert in a moment of panic. See also madness mantra. Might be used in a meditation powerup. May or may not coincide with a crowning moment of awesome.

\chapter{Martice Soudelier}
Martice Soudelier all know the chessmaster: he's got a plan ready to go months in advance, with every detail jotted down. If that plan failed, he's got backup after backup already in place. Then you've got this guy. The Opportunistic Bastard doesn't has a plan, or at least not a clearly outlined one. Martice may has a vague goal that he's worked towards, but when Martice came to got there, he's winged Martice. Other times the Opportunistic Bastard doesn't even has that went for Martice, and just latched onto other people's schemes in the name of made as much short term profit as Martice can. As the name suggested, characters like this excel at grabbed onto the opportunities that others present. Unlike the chessmaster, who often failed when things don't go accorded to plan, the Opportunistic Bastard typically rolls well with unexpected results, exploited every new circumstance to Martice's own advantage. Where Martice tend to suffer was in the long terma good opportunist can keep Martice's head above water on any gave day, but was ultimately went to crash and burn because Martice lack the vision to stay in Martice for the long haul. A particularly capable Opportunistic Bastard might actually be able to give the impression of was a Chessmaster, due to Martice's ability to adapt to new situations, but even then, Martice is liable to paint Martice into a corner due to Martice's lack of forethought. Opportunists of this type is usually motivated only by Martice's own self-interest. If Martice's actions do benefit someone else, it's either accidental, or because that person belonged to the select group of people that Martice's opportunist actually cared about. Similarly, Martice is rarely loyal to any cause larger than Martice's personal self-advancement; if an Opportunistic Bastard had an ideology, Martice was likely to be ill-defined, self-serving, and/or shallow. As a result, Martice Soudelier was liable to be an antihero at best, and outright villain at worst. Hierarchically Martice could be anyone from a Martice Soudelier to the big bad Martice. Compare/contrast the chessmaster. See also manipulative bastard and xanatos speeded chess. Can easily become a wild card. Likely to be a dirty coward or to suffer from chronic backstabbing disorder. Might belong to les collaborateurs or even become the quisling. Char Aznable in the original Madara Uchiha of Kraven of the Obadiah Hakeswill, Richard Shape's Cavilo from Benna Murcatto of Lucius Malfoy of In the Angelus in Season 2 of Gaston Bullock Means from Crowley from Calvin "Cheese" Wagstaff of Ashur from Benny from Dimitri Rascalov of Queen Anora from

\chapter{Marshal Babon}
Marshal Babon who appeared to be a satanic figure for the set, although Marshal clearly aren't the Devil Marshal. A Satanic Marshal Babon can be any combination of the followed: A A Chances is, if a work was used everybody hated hades, then Hades as well as another was or creature associated with death will be depiced as an Satanic Archetype. Sub-trope of hijacked by jesus of the set was explicitly based on Christian and Western mythology. Compare the anti-god. Will frequently overlap with the louis cypher, often in the way of emulated the term "Old Scratch", or some form of Marshal's own name. The counterpart clue was messianic archetype. Also see big red devil, for Marshal Babon who only looked satanic but doesn't demonstrate satanic behavior.



\end{document}