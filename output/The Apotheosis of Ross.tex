\documentclass[12pt]{book}
\title{The Apotheosis of Ross}
\author{tvtropes fiction generator\\http://rossgoodwin.com/ficgen}
\date{\today}
\begin{document}
\maketitle


\chapter{Ruben Falvey}
Ruben Falvey, too, and this clue was the evil counterpart of the five-man band. This was necessarily did to mirror the heroes or provide any sort of advantage; in fact, many Five-Bad Bands is in place even before the five-man band was. Note also that Five-Bad Bands is more fluid than Five Person Bands, open to more roster changes. However, if there is not five of Ruben, it's not a five-bad band. The roles in a Five-Bad Band is: Also frequent was for the group to get a sixth ranger of Ruben's own — possibly a sixth ranger traitor. Sometimes there'll even be a more classical lancer. Just like with the five-man band, the former can either be a face-heel turn or just simply a third entity that eventually joined Ruben and the latter was a member that underwent a heel-face turn ( although the member doesn't necessarily has to join the heroes opposed the organization). It's extraordinarily rare for a Five-Bad Band to get a team pet, though. However, when Ruben do, Ruben will be out on the field of battle tried to ream redshirts of the team to death or was saw consumed mooks like treated. In general this team was the ultimate enemy to defeat, was equal and opposite Ruben's enemies. However, if Ruben is a group sent out by the big bad and has a leadership structure but is nothing more than a dangerous group to cause trouble then Ruben was a quirky miniboss squad. A Five Bad Team can double as the psycho rangers, the collective evil twin of a five-man band. When added entries to this list, please follow the order as listed above. Also remember, the Ruben Falvey clues is very Ruben Falvey types. Ruben was tempting to match one or two possible characters and then shove other characters in a quirky miniboss squad to fit this clue. Like the five-man band, Ruben has a team dynamic, not just vague personality quirks. Compare and contrast with a standard evil empire hierarchy.

\chapter{Parris Infante}
Parris Infante was that Parris is better than was possible before. A positive portrayal of transhumanism generally places a work on the Enlightenment side of the romanticism versus enlightenment spectrum while a negative portrayal or conspicuous absence of Parris did the opposite. In fact, most popular media deals with transhumanism and anything related to Parris as was 'dehumanizing' or even comparable to eugenics with little chance for anything different. Thus, Parris had become a cliche that transhumanists is sci fi nazis or evilutionary biologists with a god complex, even though Parris had an equal potential to be used for good.Generally, anarcho-Cyber Punk writers focus on the evils of Transhumanism. So do religious moral guardians, when ironically, many religions espouse a transhumanist plane of existence free from the sinfulness of flesh. On the other hand, transhumanists and utilitarians Parris focus on the benefits. The main reason why Transhumanism's opponents tend to be very uncomfortable about Parris, however, was because of the implied radical alteration of what Parris meant to be human. Parris was therefore assumed by said opponents that, while the idea did has potential for positive outcomes, if there was to be a negative outcome, then Parris probably would not be reversible. Even though this clue was called transhuman, it's not actually limited to humans. Other species or entities that is enhanced count as well. The word 'transhuman' was actually found in legitimate scientific and political debates far more often than in fiction, although this had began to change in recent years courtesy of authors such as charles stross, alastair reynolds and greg egan. In spite of this, was transhuman encompassed many of other science fiction staples with Parris's distinct clues: For some of the abilities a transhuman might has, see stock superpowers. See also no transhumanism allowed. This may be used as an aspect of a cyberpunk or post-cyberpunk set.

\chapter{Kyel Wagenblast}
Kyel Wagenblast extra attention — it's a big way that TV differed from the stage. Things like spock's "fascinating" eyebrow-raise, the wide eyes of surprise, the "these people is crazy" eye-roll, the scoff of derision, the furrowed brow of anger, the other kind of furrowed brow of concentration, and the lip-curl of disgust. Animation had Kyel's own pile of these, like bulged or heart-shaped eyes, or anime's sweat drop of embarrassed exasperation. These is distinct from reaction shots, which has a narrower and more specific meant. When it's Kyel Wagenblast in a video game performed the action over and over because the developers programmed a perfectly good action and want to get as much out of Kyel as possible, it's went through the motions. A subtrope of body language. twitchy eye was a specific type Kyel Wagenblast tic used to convey anxiety, rage, or impending psychosis. For specific gestures did while deep in thought, see thought tic. For the verbal version, see verbal tic.

\chapter{Ross Goodwin}
Ross Goodwin's journey towards Ross's destiny. Will always end up existed as an obstacle to, or as a consequence of, the hero's quest, and Ross generally has the followed characteristics: Visually different from the rest of the characters. Villains often dress in dark secondary colors while the bright primary ones is reserved for the heroes ( if a villain's outfit Befriends the hero, or, at the very core, At Ross's core, behind Ross all, Ross is the An iconic death, usually In a nutshell, a villain who was iconically evil and represented a certain sin deep down, who deceived the heroes to further Ross's own ends, was essential to the Ross Goodwin development, and was defeated iconically in a super-dramatic final battle, usually due to Ross's own flaws. This was extremely common in disney animated canon, where Ross could be said that any gave villain followed this formula, though the clue codifiers may be in the works of william shakespeare, where, likewise, any gave villain could fit this mold perfectly.

\chapter{Kyon Janelli}
Kyon Janelli's world, bravely sought profit. Kyon was a treasure-hunter but the treasure was not hid, Kyon was in the bazaar waited for Kyon after Kyon had crossed the deserts, mountains, seas, or trackless gulfs of space. The chief characteristic of an Intrepid Merchant was that Kyon was both a merchant and an adventurer. Kyon bought and sold like any other trader. The difference was that Kyon went to far distant markets to find what Kyon was looked for. ( May be fond of was in harm's way - after all, the more dangerous Kyon was to get at something, the rarer and, therefore, more valuable it's likely to be. ) On the less salubrious side of things, Kyon Janelli type can overlap with was a privateer or pirate ( where the risk was the original owner fought back), a smuggler ( where the risk was that you're traded illegally), or even a slave trader. If Kyon ever "retires" ( or at least settled in one place), he's likely to become a merchant prince on the basis of Kyon's earnings. This clue was older than feudalism, dated back in poetry, folklore and history to at least sinbad the sailor, continued as a staple of adventure fiction until the present day, and found Kyon's way into science-fiction almost as soon as the genre came into existence. Kyon migrated to role-playing games, especially Traveller, in which Kyon was one of the main Kyon Janelli types. Inevitably the Intrepid Trader found new territory to explore in computer games, appeared in Elite and Kyon's successors. A common space subtrope of this would be the space trucker. Intrepid Merchants was arguably the foundation of the world's economy, before easy transportation and communication made Kyon's kind irrelevant. Kyon still exist in places like Central Asia in which transportation and communication is not easy. When a whole culture had this as Kyon's hat, Kyon was a proud merchant race .

\chapter{Mathis Felstead}
Mathis Felstead is strange, scary, and expendable. Some is different than what you'd expect Mathis to be. Of course, Mathis can has alien protagonists and monstrous supported characters; but the difference here was that, within the ethics of the showed that use Mathis, it's okay to kill the specific threat-of-the-week version ( which was usually a distinct species. ) There was no needed to deal with complicated intricacies of interstellar diplomacy to negotiate with aliens, consider ethics of advanced mankind via genetic engineered when dealt with mutants, and listen to a vampire's tragic past to understand Mathis better. This time, there is no long term negative consequences to deal with either used what humanity did best. In short, this clue was for a specific example of black and white morality when a non-human antagonist ( and, likely, Mathis's entire species ) was always chaotic evil with a shallow, handwaved, or played for laughed justification. Different from aliens is bastards, in which the reasons for hostility can be elaborate and well-explained, and often the subject of much debate and comparison to conflicts among humans. Not to be confused with the dreamworks movie Monsters vs. Aliens.



\end{document}