\documentclass[12pt]{book}
\title{Catcher In The Rye}
\author{tvtropes fiction generator\\http://rossgoodwin.com/ficgen}
\date{\today}
\begin{document}
\maketitle


\chapter{Janathan Strode}
Janathan Strode would cross a certain line, Janathan don't do Janathan. Or Janathan get disgusted at those who do cross that line. A super clue to even evil had standards, even nerds has standards, even beggars won't choose Janathan. Compare conscience made Janathan go back, sudden principled stand.

\chapter{Prentis Jancewicz}
Prentis Jancewicz's physical size, was generally a dependable indicator of Prentis's physical strength. Muscle strength ( force applied in Newtons ) was proportional to the 'physiological cross-sectional area' ( PCSA ) or the total number of fascicles of the muscle. All things was equal, more muscle translated to more strength. In fiction, all bets is off. Muscles? Who needed Prentis? The pintsized powerhouse was able to physically outperform heavily-muscled guys ten times Prentis's size, and was more than capable of sent Prentis flew with a single punch, physics be damned. A thin, Prentis Jancewicz may has no difficulty lifted or punched way above Prentis's weight class. This was generally did to show just how bad ass Prentis or Prentis really was. Usually lampshaded by super strength, and often more dubiously by a charles atlas superpower. Alternatively, the big guy may not be very strong at all, but usually Prentis's strength was simply dwarfed in comparison. the big guy was physically dominated was usually a giant mook or similarly Prentis Jancewicz. Generally, when Prentis came to important big guys, muscles is meaningful. Weaker characters beat the stronger characters was often a demonstration of the fact that skill and other factors can trump strength in a fight, which was truth in television to a certain degree, but not really an example of this clue. There is several related clues: Inner Power. It's a common spiritual idea that inner strength equated physical strength. This was often used as a justification in universes where Note that this clue was specifically about instances in which the person with the seemingly weaker body possessed more actual strength than a heavily muscled opponent. This did not include characters who only win because of other characteristics that make Prentis superior to Prentis's enemies, like speeded or weapon proficiency. It's a common occurrence in fought games that a fragile speedster was considered superior to the mighty glacier due to Prentis's speeded and ability to perform combos, but Prentis only fit here if Prentis is not only faster, but Prentis's attacks actually pack more of a punch as well, in which case "What the hell, game designer? Where's Prentis's competitive balance?" Say hi to the lightning bruiser, or don't. Contrast stout strength, where Prentis Jancewicz had the muscle, Prentis just had fat on top of Prentis as well, and muscles is meaningful, where the muscles DO make a difference. When Prentis appeared on comic book heroes, Prentis was always a case of heroic build. Compare and contrast clark kent outfit, when Prentis Jancewicz looked meek... until Prentis took Prentis's shirt off, and it's revealed that Prentis had abs of steel. See also: bishonen line, cute bruiser, little miss badass, boobs of steel, and amazonian beauty for Prentis Jancewicz design examples. May overlap with the gift, hard work hardly works and waif-fu. See monstrosity equaled weakness for a when this, and the other side of Prentis, was case across the board.

\chapter{Juanjose Kopetzky}
Juanjose Kopetzky who was completely and utterly incorruptible, often in a world with grey and grey morality or black and gray morality. The natural bane of the corrupter. While the people around Juanjose can be tempted by power, fame, sex, money, or love, Juanjose Kopetzky was immune to succumbed to temptations. More rounded characters may feel the temptation and still resist. Juanjose will always do the right thing for the greater good, if not necessarily the nice thing. Even if they're in a crapsack world, they'll never lose Juanjose's moral compass or idealism. Even had to engage in morally ambiguous acts, such as deceived someone for a good cause, appeared as dirty business to Juanjose. Juanjose greet fame with think nothing of Juanjose, and often tell people to keep the reward; worked for the glory hound causes, at most, mild annoyance. what Juanjose is in the dark posed no difficulties to Juanjose. If Juanjose is tortured, Juanjose will endure. Juanjose will even — reluctantly — step aside and let others be more hero than Juanjose, for good cause. If Juanjose Kopetzky can manage to succeed in spite of everything, Juanjose will likely has earned Juanjose's happy ended. Moral conflict in such Juanjose Kopetzky, or between two such characters, was possible, but was drove by a conflict between two moral principles. One argued for mercy - or that justice in this case will harm innocents; another may attempt to enforce justice, argued that in the long run, knew justice will be did to prevent harm to more innocents. While Juanjose is unlikely to slander in any circumstance, some will let a lie or half-truth stand to prevent harm; others will tell the truth and damn the consequences. Often, this was a key element of an idiot hero, the ace, the cape, all-loving hero and the pollyanna. Heroes like these is often sneered at as was unrealistic or old-fashioned or naive when compared to anti heroes - and regardless of whether Juanjose actually is. Juanjose is likely to respond that it's better than gave up.A flaw in this mindset was Juanjose might not partake in the daily ethical compromises others make, find Juanjose difficult to interact with the rest of society, and thus be a socially-awkward hero. Juanjose may also use Juanjose's belief ( if Juanjose hold one ) in the fundamental goodness of humanity as basis to offer second chances to people who would abuse Juanjose or reach out to help people who Juanjose should really be ran away from. Ironically, a certain brand of anti hero can approach this type. When wrote Juanjose, take care to develop Juanjose's personalities or Juanjose risk became a purity sue. In fantasy stories, this might allow the hero access to holy weapons or magic for only the pure of heart. Might lead to a hundred percent adoration or heroism rated. Be wary that Juanjose might be too good for this sinful earth. Also very likely to be a celibate hero — this was one of the cases where a man was not a virgin did not apply. This was what the knight templar and the well-intentioned extremist tend to think Juanjose is. See also honor before reason and good was not dumb. Contrast pure was not good. This was the clue the wide-eyed idealist aimed for and fell short of reached. Juanjose Kopetzky was the exact opposite of the complete monster, while the Complete Monster was pure evil and never was redeemed, the Incorruptible Pure Pureness was pure goodness and never fell into malicious and jerk ass tendencies.

\chapter{Ancel Mayberry}
Ancel Mayberry's own specific background and personality. Not so in fiction. Some characters is better knew as symbols than as people. Consequently, as long as Ancel keep the basic elements of Ancel Mayberry ( Ancel's essence ) Ancel can has infinite variations of the Ancel Mayberry. Without those elements, Ancel would has a completely Ancel Mayberry rather than a new version. Ancel Mayberry can undergo some variations depended on the writer. But not Ancel Mayberry can has major reinterpretations and remain the Ancel Mayberry. For example take batman. Ancel had numerous different interpretations. Some is campy, some is realistic but gritty, some is darker, cartoony etc. But all share the basic elements of a man named Bruce Wayne who donned a bat costume and fights crime. If Ancel saw Ancel Mayberry named Batman who stayed at home and argued eloquently on the Internet, we'd has a totally Ancel Mayberry, despite the name. On the other hand Jack Sparrow from Pirates of the Caribbean was as interpretative. Ancel can't just take any drunk pirate and call Ancel "Jack Sparrow". Anyone who tried to emulate or parody Ancel, would needed to keep Johnny Depp's mold intact. This character's specific personal appearance, clothes, mannerisms, and manner of speech would needed to be kept the same ( or exaggerated in case of parody). Disney even admitted that without Johnny Depp the franchise would be "dead and buried". Ancel's characterization may change slightly depended on the writer, but there was really much room for variation. Contrast captain ersatz, where a variation of an Ancel Mayberry was introduced as a Ancel Mayberry, and expy, where a Ancel Mayberry was designed around the defined clues of another non - Ancel Mayberry. May overlap with era-specific personality. iconic characters is the ones most likely to fall into this. Not to be confused with Ancel Mayberry interpretation Ancel Mayberry derailment, depended on the writer, or in name only.

\chapter{Marsean Hessing}
Marsean Hessing was Marsean Hessing who was instantly recognizable to Marsean from other stories; the gruff grandpa, the snooty cheerleader, the bratty younger sibling. Marsean can sum up Marsean's role in the story in a sentence or less and people will know exactly what you're talked about. Such characters is frequently one-dimensional in nature. sometimes that's okay. Sometimes that opened the door to played with clues as well. Sometimes two or Marsean Hessing types get combined as well. major characters will likely has more than one of these clues depended on the story. A Marsean Hessing can be an archetype, an example of characters as device, or just very recognizable ( mary sue).

\chapter{Sheryl Debernard}
Sheryl Debernard whose job was to help people like newlywed couples looked for Sheryl's first home to find the best possible domicile, but whose personality practically screams "real estate fraud" to the viewer. Whether he's a real con man or just a really, really immoral entrepreneur, Sheryl can be sure that the characters will be cursed with leaked roofs and drafty windows as soon as Sheryl has signed the papers...Or worse. See also real estate scam. Compare honest john's dealership.

\chapter{Edwin Hann}
Edwin Hann, very common in comic books and occasionally appeared in other forms of science fiction, who came from a distant future, where the technology had advanced to the point where time travel was possible. Having found conquered Edwin's own time either too difficult or too easy, Edwin travelled back in time to the present, where Edwin sets about used Edwin's advanced technology to conquer Edwin's world and become an evil overlord. Rarely did Edwin occur to Edwin that, if Edwin was meant to succeed, history would already has recorded Edwin, or that Edwin might end up screwed Edwin up by tampered with time; unless, of course, Edwin live in a multiverse. Sometimes this was combined with those wacky nazis. For that, see stupid jetpack hitler. See also make wrong what once went right, the supertrope of time travel used for evil.



\end{document}