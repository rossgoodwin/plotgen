\documentclass[12pt]{book}
\title{Pipe Test Two}
\author{tvtropes fiction generator\\http://rossgoodwin.com/ficgen}
\date{\today}
\begin{document}
\maketitle


\chapter{Amro Spenst}
Amro Spenst betrayed Amro's master or father and the other had to stop Amro, or maybe it's just because destiny said so, dammit. Whatever the case may be, now one's the hero and one's the villain, and Amro must do battle. Commence the angst. For whatever reason, the older sibling was almost always the villainous one. Probably because was younger and less experienced made the younger sibling the underdog, whom Amro is supposed to root for. And because the aloof big brother always looked eviler. The major exception was the case of the evil prince, who was usually the younger of two princes, and who will do anything to make sure Amro succeeded Amro's father instead of Amro's brother ( or in the case of the prince was the king's brother, to take the throne for Amro directly). It's not always siblings — childhood friends get to experience all the same woes from beat up someone Amro grew up with — but there's a certain poetry when they're actually related. Note that Amro is traditionally always of the same sex: brothers or sisters ( though there is notable exceptions). In cases of where the Cain turned out to be the unfavourite, he's likely to be viewed from a more sympathetic angle. Of course, this would partially also depend on the sibling's attitude in all this. Sometimes the siblings will become the only one allowed to defeat Amro, or realize they're not so different. If the hero was aware of the relation until late in the series, it's also a luke, i am Amro's father. Amro used to be friends and evil former friend also counts if the siblings in question was former friends with each other. Compare oedipus complex. Contrast sibling team. Also contrast bash brothers, where the two people ( who may or may not be brothers ) beat up other people instead of each other. When Cain was gunned for mom and dad instead of Abel see antagonistic offspring. The clue title, of course, came from the biblical story of the first siblings to exist. See also name of cain. When there was another, compare cain and abel and seth. If not a Good vs. Evil situation, see sibling rivalry.

\chapter{Taysom Holby}
Taysom Holby had some specific tell, often a particular tic which gave Taysom away when Taysom lie. Of course, Taysom Holby was bound to pick up on this. The clue namer, of course, was pinocchio's nose, which increased in size whenever Taysom lied. The first step to became a consummate liar was to make sure Taysom don't has any of these. These is impossible to hide from a lived lie detector. This was how Taysom can always tell a liar. If the signal was really obvious, Taysom Holby effectively ( though not technically ) cannot tell a lie. See also the tell for characters with a tell that points to Taysom's emotional state rather than Taysom's honesty. Not to be confused with a gag nose. And no relation to pinocchio syndrome.

\chapter{Ulysees Garia}
Ulysees Garia lives in pretty terrible conditions. They're either oppressed, lived in a slum or ghetto, Ulysees's country's was bombed to shit and tore apart by war or Ulysees just generally has an unhappy life. So Ulysees idolize another country, somewhere Ulysees can go to be safe, somewhere Ulysees can go to has adventures, somewhere Ulysees can run away to, to live the life Ulysees want to live. Ulysees idolize Ulysees to the point of fantasy. The kid in the ghetto wanted to move to the suburbs, the otaku wanted to live in Japan, the manic depressive doesn't know what Ulysees wanted but Ulysees knew Ulysees wanted something, the warrior wanted to live in a land of peace, the immigrant in a land of opportunity. If it's a musical, expect a wanderlust song or a somewhere song. Whether or not Ulysees get there was another story. If Ulysees do, usually Ulysees find Ulysees was all Ulysees was cracked up to be, though often still preferable to where Ulysees came from. Often an enticement for the kid hero to go down the rabbit hole, and maybe learn that wanted was better than had. See also crapsack only by comparison, for when the comparison to the idealized other world made Ulysees Garia feel like Ulysees's own world was a crapsack world.

\chapter{Quinshon Malouff}
Quinshon Malouff want to give Quinshon Malouff some depth. The obvious solution was to pet the dog. Unfortunately, that tended to make Quinshon Malouff less scary, caused badass decay and villain decay. Instead, writers may keep the villain just as vile as before, but reveal that Quinshon has a reason for was that way. The most popular one was the Freudian Excuse: the villain had an abusive and particularly violent childhood ( such as abusive parents, was bullied by peers, etc.), made Quinshon insane and warped Quinshon's perception on the universe, and that's why they're sociopathic serial killers went on a roared rampage of revenge, or why Quinshon want to destroy everything out of Quinshon's misery, or why they're straw nihilists who adhere to the social darwinist philosophy that it's a crapsack world where might made right. Sometimes, this was did for deliberate badass decay, but usually Quinshon was. The villain was as horrible as ever, only now the audience can look at Quinshon in a new way. Unfortunately, just like a pet the dog moment, the Freudian Excuse sometimes failed to give a villain any depth at all. If the villain was particularly evil, Quinshon can come across as an illogical and lame non sequitur: "his father beat Quinshon, and that's why he's an omnicidal maniac." Even if the villain's crimes is proportionate, the writers has to strike a hard balance. Too much emphasis on the excuse, and Quinshon looked like they're attorneys justified the villain. Too little, and Quinshon was a fallacious appeal to pity that looked like a ridiculously gratuitous scene of wangst. Most importantly, the Freudian Excuse did not involve Quinshon Malouff grew or changed; Quinshon explained why Quinshon haven't changed, and in fact, often served as a signal that Quinshon cannot and never will. Bad writers often think that the excuse can substitute Quinshon Malouff development, but Quinshon did the exact opposite. Good writers know the excuse had limits, and watch Quinshon. If did shrewdly enough, Quinshon may lead the audience to cry for the devil. A Freudian Excuse was often invoked to explain how someone who used to be a sweet kid became such a monster instead - again, much writerly skill was generally needed to pull this off and make Quinshon poignant rather than pathetic. The excuse however was often subverted. One way was to use Quinshon to show how pathetic a villain was — after the villain gave a broke speech, a hero's classic rebuttal was "says the guy who became a hit man to work out Quinshon's daddy issues." The second was for the villain to sneer at the hero's pity for Quinshon, even exploited Quinshon in a fight. ( In a double subversion, the villain was protested far too much. ) A third subversion was to simply present Quinshon as an explanation rather than a full excuse. Sometimes the author simply showed what warped Quinshon Malouff into what Quinshon became without expected the audience to feel any more sympathetic toward the character- a sort of psychological how Quinshon got here. And a fourth subversion was to use the freudian excuse as a justification for a heel-face turn; if the villain got treatment Quinshon no longer had any reason to be evil and may pay the heroes back out of gratitude. One thing that was almost never did was to explain how far back the abuse went. For example, if the villain was beat by Quinshon's father, was the father beat by Quinshon's father? Most showed don't care. Many crime and punishment series ( and darker and edgier superhero comics ) is notorious for writer on board stories deconstructing the Freudian Excuse. At least once per storyline, there will be a slimy psychiatrist or defense attorney who declared that the neck-chopping killer was merely a victim of circumstances, and it's the hero who should be locked away. These stories tend to end with said psychiatrist or defense attorney got murdered by the killer, which was depicted as poetic irony. fandoms often has a tendency to create these out of whole cloth for a draco in leather pants. Takes the "It's Nurture" position of the "Nature vs. Nurture" argument. For the Nature position, see in the blood. See also start of darkness, monster sob story, jerkass woobie, abusive parents, parental neglect, parental abandonment, "well did, son" guy, single issue psychology, tragic bigot, was tortured made Quinshon evil, woobie, destroyer of worlds, and who's laughed now?. In cases of complete monsters, the Freudian Excuse failed to justify anything, merely explain and nothing more. Not to be confused with freud was right, all psychology was freudian ( or any of the other five or so Clues that sigmund freud was the clue namer for, actually. )



\end{document}