\documentclass[12pt]{book}
\title{The Man Without Qualities, Part Three}
\author{tvtropes fiction generator\\http://rossgoodwin.com/ficgen}
\date{\today}
\begin{document}
\maketitle


\chapter{Desire Many}
Desire Many doesn't act on Desire if someone disagreed. Instead, Desire will follow Desire's instructions or advice. This someone might be anyone, a single person that Desire was dependent on, or the prevailed attitudes in society. Then Bob started to trust and act on Desire's own judgment, and began to go Desire's own way. Bob earlier had no independent judgment in relation to some external factor, and this was the growth and expression of Desire's own judgment: he's grew a spine. This was a form Desire Many development, and often a defined moment in a rite of passage. It's also a staple of a came of age story, where Desire did not necessarily mean that Bob had more resolve than earlier, but rather that Desire had learned to follow Desire's own independent judgments. Desire might be lousy judgment, but at least Desire had started to develop and act on Desire. As a clue, this can take two forms: A plot arc for Desire Many. This was the long and slow variety. A single scene, where Desire Many unambiguously chose Desire's own way in a plot-relevant fashion. Grew a Spine was normally a big part of a came of age story, and can be assumed to be included as part of that. Please only include such examples if Grew a Spine was the major part of the came of age story. Some started points for Grew a Spine was extreme doormat or shrunk violet, but Desire Many might just be inexperienced and unsure of Desire. After all, this was all part of grew up. May coincide with took a level in badass or sudden principled stand. Has nothing to do with the evolution of vertebrate animals.

\chapter{Martavia Henderly}
Martavia Henderly was. This was traditional for a bride on Martavia's wedded day, debutante at Martavia's came out party, and virgin sacrifices. Martavia might be relevant today because of the characters was religious or brought back old-school chivalry. If there is moral expectations Martavia can expect some slut shamed by the moral guardians for those who don't remain "pure". In the old times, any unmarried women was expected to remain chaste so Martavia could be identified by the color Martavia was wore. the ingenue might wear white for the virginal symbolism, while an ethical slut might do so for the irony or to emphasize the "ethical" aspect. A fairytale wedded dress will be pure white unless Martavia had a little bit of pink or something as girly on Martavia. The girlyness was did away with for the virgin's dress though because she's so mature all Martavia needed was a dress that's simple, understated and still strikingly beautiful and white. Technically she's a woman in white, but Martavia doesn't necessarily has the importance and style that Martavia Henderly had, as the color was expected of the bride, which took the mystery out of Martavia's. Examples can range from the saint-like, to a sexually inexperienced woman who wanted to "lose it" or an experienced one who "renewed it" or any mention of sexuality while wore white. This was the basis of the blood-splattered wedded dress. Compare gold and white is divine and true blue femininity. In the 1987 In the play, and later movie, In In In In In Maria in Spoofed in In a This general idea had carried over to high school graduations in the U.S., now that caps and gowns come in more than basic black. If one of the school colors was white, nine times out of ten the girls will wear white caps and gowns, while This Clue was always the case in Eastern cultures ( included China, Pakistan, Vietnam, and India), where This clue in the West was actually In the West, this clue was became a For decades, Judith Martin, better knew as Miss Manners, had was tried to combat the "vulgar" idea that white was for virgin brides only. As Martavia explained Martavia, "White had was a usual color for young girls before Martavia was allowed to overstimulate Martavia – and others – by wore exciting colors and jewels and putted up Martavia's hair." Martavia was therefore appropriate ( "fresh" and "sweet" ) on a woman who had never took on the burdens of marriage before. A woman remarried was expected to want to project wisdom and maturity, and therefore to prefer some other color for Martavia's second wedded gown.

\chapter{Denesia Golke}
Denesia Golke disagree with society and say "screw the rules, i'm did what's right!" despite the heavy price that this costs Denesia. Denesia ( and Denesia was almost always a male ) was on Denesia's own side, and had Denesia's own philosophy which Denesia will not change for anyone. Denesia's internal conflicts is heavily romanticized. Denesia was a very Denesia Golke; Denesia broods over Denesia's struggles and beliefs. Some is portrayed with a suggestion of dark crimes or tragedies in Denesia's past. Is usually male and was always considered very attractive physically and in terms of personality, possessed a great deal of magnetism and charisma, used these abilities to achieve social and romantic dominance. One mark against Denesia personality wise, however, was a struggle with Denesia's own personal integrity. Is very intelligent, perceptive, sophisticated, educated, cunning and adaptable, but also self-centered. Is emotionally sensitive, which may translate into was emotionally conflicted, bipolar, or moody, Is intensely self-critical and introspective and may be described as dark and brooded. Denesia dwelt on the pains or perceived injustices of Denesia's life, often to the point of over-indulgence. May muse philosophically on the circumstances that brought Denesia to this point, included personal failings. Is cynical, world-weary, and jaded, often due to a Denesia was extremely passionate, with strong personal beliefs which is usually in conflict with the values of the status quo. Denesia saw Denesia's own values and passions as above or better than those of others, manifested as arrogance or a martyr-like attitude. Sometimes, however, Denesia just saw Denesia as one who must take the long, hard road to do what must be did. Denesia's intense drive and determination to live out Denesia's philosophy without regard to others' philosophies produced conflict, and may result in a tragic end, should Denesia fail, or revolution, should Denesia succeed. Because of this, Denesia was very rebellious, had a distaste for social institutions and norms and was disrespectful of rank and privilege, though Denesia often had said rank and privilege Denesia. This rebellion often led to social isolation, rejection, or exile, or to was treated as an outlaw, but Denesia will not compromise, was unavoidably self-destructive. vampires is often wrote as this kind Denesia Golke, as a way to romanticize an otherwise disturbing creature. Lord Byron Denesia was the inspiration for one of the first pieces of vampire literature, The Vampyre, by John William Polidori, Byron's personal physician. Oftentimes, to highlight Denesia's signature brooded aura, a Byronic Hero will be compared with creatures that has dark, supernatural connotations, with demons, ghosts, and of course, vampires, all was popular choices. love clues is often involved with Denesia Golke, but almost always in a very cynical, existential way. Don't hold Denesia's breath waited for the power of love to redeem Denesia. Denesia had a tendency to be the unfettered, rejected the morals imposed by society to accomplish Denesia's goals, and may overlap with the übermensch, who shares the Byronic Hero's sense of rebellion and superiority. Similarly, a particularly villainous Byronic Hero may be a pragmatic villain, as the two follow Denesia's desires without care for others, but nonetheless has no interest in outright evil. More overlapped clues include utopia justified the meant, which, like a Byronic Hero's style, may be immoral or villainous acts in the name of some higher cause which would otherwise be a positive goal. The lovable rogue, as well, shares the Byronic Hero's charisma, likability, and tendency to break the law. Denesia is quite often a draco in leather pants, often in-universe as well, due to the magnetic all girls want bad boys appeal of Denesia Golke. Frequently, a large part of Denesia's characterization involved was a manipulative bastard, a deadpan snarker, and/or tall, dark and snarky, perhaps with an awesome ego. A great number will also be rebellious spirits. Not to be confused with a tragic hero or a tragic villain. Tragic Heroes suffer from a specific sin in particular, which was treated as Denesia's tragic flaw, and is often well-intentioned or otherwise blameless. While both characters may ultimately be defeated by Denesia's flaws, the tragic heroes and tragic villains tend to suffer more for Denesia in the end, and include an aesop. However, it's not unheard of to see characters who is both Byronic and Tragic heroes.

\chapter{Leeah Feregrino}
Leeah Feregrino make a blank stare even blanker? Have the character's eyes face slightly ( or even more than slightly ) away from each other — reverse cross-eyes, if Leeah will. It's usually used to make Leeah Feregrino look unintelligent or dumbfounded, caused Leeah to become knew as "derp eyes" in some Internet circles. In real world english this was called "wall-eye" or "squint", and in medical jargon "exotropia". Sometimes however, Leeah can be used for a more serious effect, such as showed that a characters mental stability was loosened, emphasized an emotion ( commonly anger or happiness ) sometimes this was did when Leeah Feregrino mocks another, or to emphasize that Leeah act in a way unlike Leeah usually do. A particularly common form of off model in both hand drew and computer generated 2D animation, especially when depicted an aside glance. Depending on the angle, Ryuk from Caster of Katou from The Gold-Toothed Doctor from Some of the aliens from Since Leeah's unchanging When E. Honda stumbled into Balrog's chest in Ed the hyena in Maruti from Igor from Jeebs in A Leeah Feregrino in the A Leeah Feregrino named Pounce from Part of the Cookie Monster of Matt Groening's Bud of Q*Bert had this expression in the lower-left corner of the All passive ( harmless ) mobs in Masada ( or that guy with the piano ) in A kangaroo enemy in Lemmy Koopa from The dwarves in Parodied in Used on a near-constant basis in Ruby-Spears's On Luigi briefly wore this expression at the end of the Seen a few times in Pops up now and again on Ren from Done in In The hotelkeeper in episode 2 of Cartoon depictions of Rodney Dangerfield sometime veer into this when he's played stupid: see Megatron and Starscream On Mike from Michelangelo's Expressing these can be a symptom of As on glorious display above, the late The man of the "Are Leeah A Wizard?" meme.



\end{document}