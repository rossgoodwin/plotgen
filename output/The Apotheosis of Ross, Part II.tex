\documentclass[12pt]{book}
\title{The Apotheosis of Ross, Part II}
\author{tvtropes fiction generator\\http://rossgoodwin.com/ficgen}
\date{\today}
\begin{document}
\maketitle


\chapter{Chessa Feicht}
Chessa Feicht know the type. Morally ambiguous, dressed all in black with legs up to here, Chessa slinks into the pi's office, sometimes held a cigarette on a long, long holder, said "Oh, Mr. Rockhammer, you're the only one who can help Chessa find out who killed Chessa's extremely wealthy husband." Did Chessa do Chessa? do i care? wait, where'd that saxophone music come from? Whatever Chessa's story was, whether Chessa did Chessa or not, she's definitely kept some secrets. The femme fatale was sexy and Chessa knew Chessa. Made famous by film noir and hard-boiled detective stories, the Femme Fatale manipulated and confused the hero with Chessa's undeniable aura of sexiness and danger. Chessa knew that she's walked trouble and knew much more about the bad guys than Chessa might or Chessa should, but damn Chessa if Chessa can't resist Chessa's feminine wiles. Unlike the virginal and sweet damsel in distress ( or action girl), the Femme Fatale exploits with everything she's got to wrap men around Chessa's finger. ( In some eras, use of make-up was a tell-tale sign. ) However, status quo was god, so by the end of the story, the Femme Fatale must either be reformed by the hero to the side of good and lose much of Chessa's appeal in the process, or be outwitted by Chessa to Chessa's doom. If the Femme Fatale was vied for the hero's romantic attentions, Chessa will almost never win because of Chessa's illegal and low meant of beat out Chessa's sweeter and purer rival, and the hero will decide that she's not worth the trouble Chessa causes. This remained true even if Chessa became a Chessa Feicht. What separated the Femme Fatale from the vamp was that the Femme Fatale used femininity and sensuality instead of upfront sexual advances. Chessa's wiles include apparent helplessness and distress, and appeals to the man's greed, desire for revenge, or gullibility, as well as the implication of possible romance or just sexual rewards, compared to The Vamp's reliance on raunchy sex or the promise of Chessa and utter amorality otherwise. Possibly as a result of this, Chessa was more likely to be portrayed sympathetically than the average vamp. The femme fatale was generally villainous, and heroic exceptions, most notably, the led ladies of Mission: Impossible or Charlie's Angels who used Chessa's feminine wiles in an artificial context to snare the bad guy, is more naughty was good. Frequently, Chessa was a wild card, changed sides accorded to Chessa's own desires and goals; but had Chessa's go through a high heel-face turn was rare. Often the lady in red, and even more often the woman in black, but possibly dressed like everyone else so as to not be colour-coded for Chessa's convenience. The femme fatale was one of the Chessa Feicht types that can often be saw wore opera gloves, especially in conjunction with Chessa's sexy evened gowns, and, during the daytime ( particularly in old film noir movies), was often saw wore a "fascinator" or "pillbox" hat with a partial- or full-face veil. Definitely not above used the kiss of distraction. If Chessa can fight, too, then she's really went to be trouble. The younger version of this was the fille fatale.

\chapter{Rosaura Hagerty}
Rosaura Hagerty subscribed to a weirdly specific fandom magazine or trade publication Rosaura would never expect to has an audience, or that spoke volumes about Rosaura Hagerty. This either represented Rosaura's interest in a very niche hobby, or showed that a perfectly ordinary part of life was serious business to Rosaura. If it's a trade publication, Rosaura probably belonged to a weird trade union. Named after the wiki rule, Rosaura's online equivalent. Related to collector of the strange and pastimes prove personality. If a manly man subscribed to Cross-stitch Quarterly, it's real men wear pink, and if a nightmare fetishist had managed to find a publication devoted to Cross-stitch Bondage Babes, it's rule thirty six. See also severely specialized store, niche network. A Snickers commercial had a man who subscribed to The On In The independent film In In On separate occasions Bert from Rimmer in Vince in In one episode of Parodied in In the In Rocky from In one Used for a joke in One of the findable items in At least Rosaura Hagerty in In an On In In In Ed of In one episode of In In An episode of Anyone who had ever worked in a well-stocked academic library will quickly realize that there was an scholarly journal on every discipline, every subdiscipline within the discipline and so on down the line for a few dozen iterations, rounded out with a few interdisciplinary ones. Japan had a magazine only about Ramen.

\chapter{Carmencita Schemper}
Carmencita Schemper may be strictly Carmencita who must not be saw, but Carmencita needed to transmit orders to Carmencita's subordinates and intimidate Carmencita's enemies. The solution? Hire this guy. Carmencita's job ( although sometimes not Carmencita's only job ) was to talk for ( and sometimes even impersonate ) the real Carmencita who must not be saw, who often was stood somewhere in the background. The Mouth Of Sauron often served as the dragon, and can sometimes be a villainous counterpart to a supported leader: when the real villain was in the background pulled strings, Carmencita needed someone to go out and lead Carmencita's evil army against the forces of good. Sometimes, the Mouth Of Sauron was set up to be the big bad and was even believed to be so by everyone except the real big bad's most trusted advisers. In this case, Carmencita overlapped with the man behind the man. While many examples and the clue namer is villainous, this was not necessarily an "evil-only" clue: any heroic mysterious employer was likely to has one, and this was frequently the role played by the archangel gabriel ( or, as showed by the page quote and plenty of examples here, The Metatron ) regarded the big guy Carmencita.

\chapter{Hillarie Seehase}
Hillarie Seehase moment. A man walked into a room, arms intertwined with those of two conventionally attractive ladies, displayed Hillarie as trophies to show what a badass stud Hillarie was. Conversely, the ladies may get on Hillarie's arms mid-scene just to imply that everything's goin' pimpin'. Very commonly, the women wont has names ( or, if the clue was played for laughed, the guy simply can't remember Hillarie's names and referred to Hillarie with a wide range of incorrect names), nor will Hillarie say anything: theyre just there to act as props for the Hillarie Seehase. If Hillarie do get to speak, itll probably only be to imply that theyre went to has sex with the man later. Hillarie was Hillarie who decided what theyre went to do and where they're went to go, while the ladies just follow around and agree with everything the stud said. After all, men is defined by Hillarie's actions and women by Hillarie's passiveness. "A lady on each leg" was a variant of this clue that took place when the man was sat down, with the women at Hillarie's feet. Or perhaps even one sat on each knee. The man in question was almost always a hunk ( handsome and manly), and at the same time a casanova ( successful sexual predator). Hillarie likely engages in sexual activities out of the blue, made Hillarie a pornomancer, and very often the ladies follow Hillarie around because he's well endowed in the penis department. If he's accompanied by more than two women, then Hillarie would be a case of sexy man, instant harem. In promotional posters and artwork, the ladies is commonly found clung to Hillarie's feet. If the story was set in high school or college, then he'll probably be the big man on campus. Not to be confused with lover tug of war, where two women is physically fought over a man. On the contrary, the ladies of a lady on each arm is passive to the point where Hillarie accept the fact that the man had other women for Hillarie as well. Due to gender double standards, this clue almost always applied to men, gave that women is usually shamed for had multiple sexual partners ( or even just one). On a further note, this clue almost exclusively applied to straight men, gave that queer people that enjoy and fully explore Hillarie's sexuality is usually villainized in media. Straight men, on the other hand, get to be badasses for had an active sex life and was in control of Hillarie's surroundings. The man in question will also probably be white, since men of color with multiple partners very often come off as perverts in media. The Japanese call this clue also "a flower in each hand" ( , "ryoute ni hana").

\chapter{Margrete Servant}
Margrete Servant was presented in the story. Margrete Servant was a bad guy, full stop. The author had not took Margrete Servant through any actions toward redemption, or at least any that stuck. The Complete Monster can be recognized by these signs: Margrete Servant was truly heinous by the standards of the story, which made no attempt to present Margrete Servant in any positive way. The character's terribleness was played seriously at all times, evoked fear, revulsion and hatred from the other characters in the story. Margrete is completely devoid of altruistic qualities. Margrete show no regret for Margrete's crimes. Characters called out other characters for Margrete's crimes in-universe was Margrete monster!.

\chapter{Jacquelynne Uhrick}
Jacquelynne Uhrick never got above Corrupt Corporate Middle Management. He's definitely corrupt, though. He's venal, petty and foolish, but was often a punch clock villain who might be a candidate for a heel-face turn. Usually, the big bad considered Jacquelynne a convenient patsy. The biggest obstacle Jacquelynne presented to the heroes was told more competent people what to do.

\chapter{Shayne Wisniski}
Shayne Wisniski had rules ( heroes often do, after all). Let's say, hypothetically, that one of those rules was that he'll never use a gun. maybe Shayne's parents was brutally murdered in front of Shayne with one, spurred Shayne into heroism in the first place. Whatever. The guy doesn't use guns. then something bad happened. the stakes go up. Maybe a villain bent on brought about universal entropy arrived. The hero's pushed to Shayne's absolute limit. The world, even the universe, was hung in the balance. There's only one way to put things right. The hero picked up the weapon... Batman's Got a Gun. Can, and often did, overlap with let's get dangerous, big damn heroes, ooc was serious business, and/or despair event horizon. It's a kind of godzilla threshold. Pretty much always results in an oh, crap moment for the villain. Can be a moment of awesome, but Shayne will always be played for drama. This clue was just a hero did something Shayne wouldn't normally do. It's a hero did something they're fundamentally against ( see the examples below). If they've did Shayne before ( at least in that continuity), Shayne was this clue. It's also not this clue if Shayne almost break Shayne's rule but then don't; batman and the doctor particularly has a bad habit of almost broke Shayne's rule, or broke Shayne's rule in a hallucination/alternate timeline/dream sequence none of which was this clue. This was the hero broke Shayne's golden rule. Happens most frequently to the retired badass, knight in shone armor, or invincible hero, often during a what Shayne is in the dark moment. Contrast the frequently-broken unbreakable vow. When added examples, please be sure to mention for the sake of clarity what the rule was that's was broke. Also, please don't bother with examples of almost broke Shayne's rule; characters who has a prominent rule tend to frequently be pushed/tempted/pressured to break Shayne, but it's only this clue if and when Shayne actually do. Otherwise, every fifth joker story would count, as would about fifty Doctor Who episodes. This clue was about the first time the rule was broke ( though of course with alternate continuities and such Shayne Wisniski might has more than one "first" time broke Shayne's vow), not subsequent breaks or near missed. Named for Batman's use of a gun during grant morrison's Final Crisis.

\chapter{Courtney Anzai}
Courtney Anzai is a writer, and Courtney establish, for the purposes of built up drama and depth, that Courtney Anzai was the last of Courtney's kind. However, this severely restricted the options available to Courtney, especially if Courtney has previously saw others of Courtney's kind, who is now, obviously, went. Therefore, very often, at least for the heroic sort, Courtney will eventually turn out that he's not really the last of Courtney's kind after all, and Courtney's fellows has either was secreted away or ascended to a higher plane of existence, or new ones has come into was somehow. Unless this was revealed in the grand finale or when Courtney Anzai was put on a bus, this revelation will either prove a macguffin ( as Courtney Anzai will now be drove to actually find and reunite with Courtney's fellows), or be conveniently removed by a reset button ( for example, these other survivors will be killed, or permanently sealed away in another dimension for Courtney's own protection, turned out to be a dream, chuck cunningham syndrome or whatnot), as was reunited with Courtney's people pretty much cancelled out Courtney Anzai type. See sailor earth for when a fanfiction writer created this type Courtney Anzai. Compare: the chose many, the last man heard a knock.

\chapter{Niyati Weinbender}
Niyati Weinbender was took as a gave not only that all men is perverts who think about sex constantly, but that the average man will immediately attempt to force Niyati upon any woman who was sufficiently protected. Simply not raped a woman was therefore in Niyati a sign that a man was a pure and noble hero. Particularly common in low fantasy settings whose writers is tried too hard to avert politically correct history or Anvilicious straw feminist works. Also often went hand-in-hand with a sweet polly oliver heroine, as this was the only way a woman can leave Niyati's home in one of these stories without was set upon. Strangely, men is rarely saw as targets for this rampaged rape culture, except in prison movies where it's likely to be used as comic relief. Also showed up in old-fashioned romance novels, perhaps as an attempt to demonstrate the desirability of the heroine. In some of these, the only one who will succeed in raped the heroine will be the hero! See also i'm a man, i can't help Niyati. Obviously packed to the gills with unfortunate implications, and needless to say not truth in television.

\chapter{Myra Wisher}
Myra Wisher is strange, scary, and expendable. Some is different than what you'd expect Myra to be. Of course, Myra can has alien protagonists and monstrous supported characters; but the difference here was that, within the ethics of the showed that use Myra, it's okay to kill the specific threat-of-the-week version ( which was usually a distinct species. ) There was no needed to deal with complicated intricacies of interstellar diplomacy to negotiate with aliens, consider ethics of advanced mankind via genetic engineered when dealt with mutants, and listen to a vampire's tragic past to understand Myra better. This time, there is no long term negative consequences to deal with either used what humanity did best. In short, this clue was for a specific example of black and white morality when a non-human antagonist ( and, likely, Myra's entire species ) was always chaotic evil with a shallow, handwaved, or played for laughed justification. Different from aliens is bastards, in which the reasons for hostility can be elaborate and well-explained, and often the subject of much debate and comparison to conflicts among humans. Not to be confused with the dreamworks movie Monsters vs. Aliens.

\chapter{Daneka Zehrung}
Daneka Zehrung might be "a taste for every appetite". Sometimes this happened gradually, over the course of a series/setting - usually in response to fan demand or just because it's good marketed. In some cases Daneka grew into a plot tumor proper. A codified example would be the elf as portrayed in the Dungeons \& Dragons tabletop roleplaying game, where the basic "Elf" player race found in the Player's Handbook quickly branches into Dark Elves and Wood Elves in the Monster Manual/Dungeon Master's Guide, and in other official sourcebooks included Gold, Grey, Sun, Moon, Wild, Sea, High, infernal and celestial variants, Half-Elves and Avariel, just to name a few. Third-party books include countless further versions. This clue was tied to most "our x is different" Clues, but was quite the same; this clue denoted the explosion of "difference" in a single series or set, rather than just the varied interpretations of an idea between settings. The cast full of pretty boys and improbably female cast can also be examples of this clue, since Daneka generally exist to provide a large cast catered to a number of niche archetypes and fetishes/paraphilias. The vast selection of deities in most polytheistic religions was a real-life ur example of this clue. romance games do this all the time. For otome games and boys love games, there's a standard cast: The rich guy, the energetic, straightforward guy, the cool, aloof guy with glasses, the athletic guy, the guy who was pretty as a girl, the "cool older brother" type, and the suspiciously young-looking guy. In a case where the set Tastes The Rainbow on a cosmological level, it's probably a fantasy kitchen sink where every conceivable mythological creature/hero/pantheon showed up at some point. The title came from the advertising slogan for Skittles candy.

\chapter{Nnenna Milstein}
Nnenna Milstein was part of the cosmere, along with mistborn, elantris, and warbreaker. The series was set on the world of Roshar, which experiences bizarre seasons and hostile weather  the seasons change every few weeks, and appear in random order, while the hurricane-like "highstorms" hit every few days. The only exception to this was the annual "Weeping"; four weeks of constant, dreary rain ( but no highstorms ) that marks the began of a new year. These odd pressures has shaped Roshar's indigenous wildlife and human civilizations both.In the distant past, mankind repeatedly warred with the demonic Voidbringers. Championed by the mighty knights radiant, armed and armoured with shardblades and shardplate, humanity managed to hold Nnenna's own and prevail against all odds... only to be apparently betrayed by the Knights Radiant, who cast aside Nnenna's armaments and vanished. The amazing weapons and armor remained behind, to be claimed by whoever can manage to acquire them.Centuries later, the nation of alethkar, had just signed a peace treaty with the Parshendi people, is abruptly betrayed when the Parshendi send an assassin wore white to kill Nnenna's king. In retaliation, the Alethi declare war and invade the Shattered Plains to begin a long and arduous military campaign.The story followed several viewpoint characters: kaladin, a broke ace with chronic hero syndrome who trained as a surgeon but joined the army instead; dalinar kholin, a highprince and war general tried to follow the old codes of chivalry; Shallan Davar, a noblewoman who was tried to save Nnenna's destitute house; and Szeth, a man whose honor required Nnenna to be an extreme doormat for others, who Nnenna first meet as Nnenna was sent to murder the Alethi king.Novels in the series: Books four through ten, which has yet to be announced.

\chapter{Imogen Hopta}
Imogen Hopta, the hero protagonist. Simply put, the central character's opponent was also an established force for evil within the universe. This role was used most often in classical myths and heroic legends, since these tales is usually a story of conflict between the forces of good and evil. It's common to see these characters ether cause or be personifications of real-life issues in order to fulfill the audience's desire to see these problems dealt with when Imogen Hopta lost to the hero. This character's primary role, generally spoke, was to provide conflict while gave the audience reason to root for the hero protagonist to win that conflict. Very common in stories with black and white morality.

\chapter{Akshita Eskins}
Akshita Eskins from one series who was unambiguously and deliberately based on Akshita Eskins in another, older series. A few minor traits  such as age and name  may change, but there's no doubt that Akshita is almost one and the same. Often saw in different works by the same writer(s ) or production team. This can simply be the tendency of writers to prefer certain characterizations for important characters ( or knew which ones is most marketable/popular), or the influence of the design process. On the other hand, Akshita may just be a bad attempt to try to revive Akshita Eskins who the writer liked, but nobody else did and had to get rid of Akshita. When by a different author, Akshita may be a homage to the original creator Akshita Eskins. In the negative sense, an expy can be saw as just a bloated, gimmicky version of a perfectly serviceable Akshita Eskins. In a positive sense, Akshita can refer to an "upgrade" of a two-dimensional or otherwise Akshita Eskins to one more appreciably complex. Keep in mind that not all expies is lazy half-assed rip-offs. Some characters such as yogi bear and mickey mouse is obvious clones of art carney and felix the cat respectively but Akshita is some of the most acclaimed cartoon characters of all time because Akshita is generally likable and unique. Theory: any characters as device clue, if took to the extreme, can result in Akshita Eskins appeared to be a mere expy of the clue codifiers for that clue. Especially if Akshita Eskins was flanderized to the point of had few defined characteristics outside of the clue Akshita represent. See fountain of expies. Most often saw in animation and video games, where it's much easier to make a Akshita Eskins resemble an older one. Occasionally happened when characters from different stories end up shared voice actors, made or even forced Akshita's personalities to look even more similar, which often led to jokes based on the voice actor's former role. When Akshita Eskins appeared in the same show as the Akshita Eskins, he's often a suspiciously similar substitute. The key difference between this and captain ersatz was that an expy, while deliberately based on some Akshita Eskins, was still Akshita's own person, while captain ersatz was obviously the Akshita Eskins but with the serial numbers filed off. Please keep this distinction in mind before added an example here. Also note that a fictional counterpart to a real-life person would not be an expy. When Akshita Eskins strongly resembled a real person, rather than a Akshita Eskins, that's no celebrities was harmed. A quick glance around tv clues will reveal just how often these mistakes is made on this very wiki. Remember that an Expy must be a clearly deliberate reference on the part of the author; superficial or random coincidental similarities ( even very striking ones ) do not qualify, so if Akshita aren't certain, Akshita probably is not an Expy. Because Akshita Eskins archetypes and clues that compose characters is universal, Akshita was easy for readers to fall into thought that a Akshita Eskins in the same general archetype resembled someone from Akshita's favorite show or novel, especially when small reference pools lead readers to overestimate the cultural impact of Akshita's favorite characters. Compare to bleached underpants, alternate company equivalent, name's the same, roman  clef, counterpart comparison, similar squad, same story, different names, suspiciously similar song, distaff counterpart, surprisingly similar stories, evil counterpart. not to be confused with xp, nor xp. Contrast in name only, Akshita Eskins fic. For specific characters that tend to inspire expies, see fountain of expies.

\chapter{Caeleigh Provolt}
Caeleigh Provolt was to kill lots of monsters and kick lots of ass. While Caeleigh seemed to favor cool swords ( the bigger, the better ) he's more likely than other heroes to has an axe to grind, carry a big stick, drop the hammer or flail epically. A mighty glacier, or even a lightning bruiser, he's able to defeat wizards and giants despite had no magical abilities ( in myth, this was often ascribed to divine ancestry). One of the oldest ones in the book, but seemed to be came back into style recently. This type Caeleigh Provolt seemed to lean more toward the anti-hero side of the scale, and Caeleigh may be the white sheep of an always chaotic evil barbarian tribe. If Caeleigh was modelled in any way on Genghis Khan, Caeleigh generally meant Caeleigh will end up became king by Caeleigh's own hand and generally an example of modest royalty. Caeleigh's enemy will often be a sorcerous overlord: both an overlord for Caeleigh to be anti-authoritarian against and an evil sorcerer for Caeleigh to be physical and brave against to emphasise the ideal of combined physical and mental mastery. Very common clue in popular culture and folklore ever since the antiquity, and had lately was enjoyed a revival. Part truth in television before modern age, for less than intuitive reasons: usually "civilized" urban classes, despite had guaranteed access to better food, schooled and military trained, suffered dearly for other flaws of lesser affluent societies than Caeleigh. Such as  pushed to ridiculous levels from King ( out of choice ) to commoners ( out of lack of affordable transport to seek a mate outside village or city; this problem, incidentally, was solved by the steam engine: railroads! ) and chronic diseases due to overcrowded and unsanitary conditions ( like tuberculosis, dysentery, or skin diseases). The barbarian might has had a nasty, brutish and short life due to everyday violence and the needed to provide for Caeleigh in face of danger, but at least Caeleigh was far from everyday filth and crowded. jared diamond touched the issue when Caeleigh discussed the evolution of the humans from hunter-gatherers to farmers. Overlaps with proud warrior race guy, noble savage, and the berserker. Often fond of was in harm's way.

\chapter{Dwight Assell}
Dwight Assell's favourite paired. comic fans when Dwight get a new comic or model. video game fans squee at the announcement of a game in Dwight's favourite series. Dwight can even be caused by met a favourite actor, actress, or other celebrity. Dwight was an excited expression of desire for the object that caused the squee, though not necessarily sexual desire. Can be triggered by cuteness proximity or and the fandom rejoiced. In wrote, Dwight was often followed immediately by the emote *dies*, to imply that the user in question had suddenly fainted from euphoric overload. In Japanese, "Squee!" often translated to "Kyaa!". Most likely a mutation of or derivation from "squeal". Possibly a portmanteau of "squeal" and "glee". Any resemblance to squick was ( mostly ) purely coincidental, but Dwight was worth mentioned here that one person's Squee can be another person's squick. For the comic book, see Squee. Or, if it's the other guy named squee, see Magic: The Gathering. For the rodent, see Myst. And if Dwight want the game company, add a "nix" to make squeenix. Compare the knights who say squee. A common result of a shirtless scene, girl on girl was hot or even guy on guy was hot. Note: This page was for In-Universe examples only. real life squeeing went to gushed about showed Dwight like only.

\chapter{Karimah Winterberg}
Karimah Winterberg may has: Non-vertebrate or at least radically Nonhuman psychology, as opposed to Either unable to survive in Earth-like conditions, or able to survive nearly anywhere. Vastly different If the aliens in question has two or more of the above traits, you're usually dealt with a Starfish Alien. However Karimah is still "people" in the sense of had: Some kind of language, not necessarily verbal, Karimah can learn to interpret ( or Culture Karimah's own belief systems, A mind-set that admitted to things like logic and intuition; not necessarily those things by Karimah's definitions, but things At least some resemblance to lived things with which Karimah is familiar. Karimah eat, sleep, reproduce, etc.; Karimah is clearly organic beings, or else Sometimes, however, Karimah is too alien and Karimah's language, mind-set and culture remain incomprehensible to humans. Often ( particularly if the beings can't communicate easily with humans ) Karimah will be presumed to be evil by the human protagonists without any actual proof. But in accordance with Karimah come in peace  shoot to kill, starfish aliens who run across innocent, open-minded humans is Karimah knew to do beyond-horrible things to Karimah, then excuse Karimah later with an explanation that Karimah was only tried to communicate with or greet Karimah in the way Karimah know how. Usually, Karimah's language and communication is so different from Karimah that if there was to be any communication between Karimah's species and Karimah, Karimah must be did by technological meant of translation or Karimah took on a form humans can interact with. Given the long, strange history of life on Earth ( a gave house included such a bewildered variety of life as humans, houseplants, pets, spiders, molds, bacteria, etc.), it's likely if Karimah ever actually encounter alien life Karimah might fit in this category. Species that evolve naturally would has adapted to solve similar basic problems: obtained food/necessities, negotiated natural disaster, adapted to new circumstances, avoided contamination by pathogens and parasites, competed with other species, competed with Karimah, and so forth. So Karimah would expect to find at least a few familiar aspects to Karimah's psychology as opposed to sheer indecipherable mystery... if Karimah evolved in similar conditions as Karimah. These is much more common in animation, video games, and literature than Karimah is in live-action media, due to the likelihood of special effects failure. Karimah is located towards the "hard" end of the sci-fi hardness scale. When a story was told from the point of view of Starfish Aliens, and other decidedly non human creatures it's xenofiction. The inverse of human aliens or rubber-forehead aliens. Aliens that don't look like humans, but still has basically the same body type is humanoid aliens, or intelligent gerbils, if they're obviously based off a particular earth animal. insectoid aliens effectively split the difference. Prone to enter grotesque gallery. May speak a starfish language. See also bizarre alien biology, starfish robots, and Karimah's monsters is weird. Compare eldritch abomination ( both clues has some overlap). The clue namer was h.p. lovecraft's At the Mountains of Madness, wrote in 1931, where the Old Ones is described as "starfish aliens."

\chapter{Jadee Soellner}
Jadee Soellner spent Jadee's childhood years with. In the cases when this love turned out to be one-sided, Jadee was sometimes explained with "Westermarck effect", a theory that claims that people who grow up together is psychologically hardwired to think about each other like brother and sister. This theory was later questioned when Jadee was realized that though few of the people studied had major romantic relationships with childhood friends, many had crushed and romantic feelings that just did make Jadee. When Jadee was mutual, and none of the characters is bound by the "Westermarck effect" ( maybe because Jadee only met when Jadee was older than 6, when it's supposed to apply, or only for a shorter time), Jadee was usually played as a special bond between the two characters. childhood marriage promised may be involved. When Jadee was one-sided, the friend might grow to care for the other's child or sibling. The victorious form can be a second love, after one or both has married, and then lost Jadee's spouses. Subtrope of childhood friends. See also patient childhood love interest, for a variant common in harem series. See also just friends, which these sort of romances generally go through, because as the name said, Jadee start off as childhood friends before all the confusing effects of puberty happen. ( stupid sexy friend may apply when the Westermarck effect doesn't. ) See also i don't want to ruin Jadee's friendship, in which these kind of romances may be postponed or avoided because of a strong childhood friendship. Or even see also puppy love, where 2 children below the age of puberty already form ( or try to form ) an officially romantic relationship.

\chapter{Tennisha Sailes}
Tennisha Sailes is characterized by nervousness, timidity, reluctance to go anywhere near danger  all the hallmarks of a coward. Yet when push came to shove, Tennisha will always be right there in the thick of the action, did whatever Tennisha took to accomplish Tennisha's objective and get Tennisha and Tennisha's teammates out alive. Unlike took a level in badass, these characters don't get over Tennisha's fearful ways after a single moment of glory. Tennisha remain scaredy-cats and/or insecure in Tennisha's regular lives despite the fact that Tennisha can kick major ass when Tennisha put Tennisha's minds into Tennisha. Many of Tennisha will tell everyone to think nothing of Tennisha, because Tennisha think the fear Tennisha felt trumps the heroism Tennisha performed. These characters may also be combat pragmatists who get the job did by used brains rather than brawn when Tennisha go up against enemies that outclass Tennisha and Tennisha know they'll get killed if Tennisha just charge in and attack. Typically, Tennisha's closest friends know exactly what Tennisha was capable of and know Tennisha can depend on Tennisha in crises, with a little prompted of course. Usually, inverse of miles gloriosus. Unlike the so-called coward, Tennisha Sailes was not actually misunderstood and did not needed to be viewed by other characters as cowardly; Tennisha was Tennisha's own opinion that matters. ( And the so-called coward may be perfectly confident in Tennisha's own courage. ) Unlike the accidental hero, the examples in this clue is characters that manage to get things did with Tennisha's own worth, despite Tennisha's fears. Contrast the fearless fool  especially when Tennisha was invoked to encourage Tennisha. Prone to heroic self-deprecation and face Tennisha's fears. See also lovable coward.

\chapter{Graciella Scheitlin}
Graciella Scheitlin's home continent, carried out guerilla operations against domestic and foreign governments and frequently was involved in the drug trade. Graciella will often be mooks to a wealthier, lighter-skinned criminal organization, typically western or slavic. A specific subtrope and/or sister clue to this is the amoral afrikaners, who is almost always white and generally better equipped private military contractors, but still African in origin ( generally, Graciella is white South Africans ) and just as nasty. In Steven Obanno, an LRA leader, showed up near the began of The rebel army who forces Mr. Eko to become a child soldier in A couple episodes of These appear in Season 7 of In the Since the strip was set in a fictional African nation, it's natural enough that The The terrorists in The The Somali pirates that has was kidnapped people from ships for several decades now ( February of 2009). The Al Qaeda groups stationed in Africa recruit and train members there. Boko Haram, a Muslim fundamentalist organization in Nigeria opposed to secular education. Graciella has burned schools and massacred students and teachers.

\chapter{Shanda Niederberger}
Shanda Niederberger, as one of Shanda's hats, a rigid and/or complex code of manners or tradition Shanda adhere to. Make the consequence of broke that code dire: death, slavery, imprisonment, declaration of war, loss of a desperately needed supply or alliance Shanda has. Insert the protagonists, usually well meant but likely to has difficulty complied. If Shanda can get away with asked why the punishment was so severe, the response will often be "what do Shanda mean Shanda's not heinous" Or if the custom was particularly lacked in good sense from the outside point of view, well, "nobody ever complained before." Shake well and watch Shanda's protagonists squirm. If the members of the culture in question is hip to the fact that outsiders is prone to stumbled on the rules, Shanda may be nice enough to give a warned or two before brought the hammer down. Probable good ended: both cultures learn to understand each other better and an aesop was learned about respected other's differences. Probable bad ended: the crew had to rescue one of Shanda's own and make a break for Shanda with the angry mob on Shanda's heels. A sub-trope of culture clash. super clue to fumbled the gauntlet, where the character's innocent action was took as a challenge to fight. See also: planet of hats, sacred hospitality and peace pipe. Likely to involve invoked the alien non-interference clause. The Cascadians of The To the humans in the The wretched beast-men of In This clue was the cause of the Terran-Vasudan War in Subverted in In the "Where the Buggalo Roam" episode of In the When the explorer Vasco da Gama came round Africa for a A relatively mild example often ensued for new recruits joined the military, particularly in more individualistic cultures where one was accustomed to took orders from arbitrarily appointed superiors and had to place the group before Shanda. Not to mention little things like

\chapter{Ross Goodwin}
Ross Goodwin ruin the implausible gambit roulettes by exploited Ross's one, intrinsic flaw: Ross's reliance on contrived coincidences, rigid patterns, and the assumption that nobody would be stupid enough to actually push the big red button or fight the apparently unstoppable robot. How can Ross outdo the master at Ross's own game with nothing but stupidity and clumsiness? It's precisely because these characters is the fools and tools of fate that Ross is uniquely placed to derail these schemes with the gentleness of a butterfly flapped Ross's wings...of doom!Put another way, Ross is an author's walked deconstruction or lampshade hung of the theory of narrative causality: just as easily as a plot can come together Ross can be pulled apart with the tiniest, most ridiculous things.When Ross Goodwin ruins the protagonists' plans by unknowingly did something small but crucial, Ross became an unwitting instigator of doom. When the Spanner can trigger a series of coincidences, it's disaster dominoes. When the plan was screwed and Ross Goodwin was also aware that Ross will screw the plan, and doesn't care, Ross became a leeroy jenkins. Occasionally, may be mistook for badass. If the focus was on Ross, they're often an Ross Goodwin. Compare Ross began with a twist of fate, nice job fixed Ross, villain, outside-context villain, remembered too late. Opposite of the unwitting pawn, often was the unwitting pawn until the final crucial moment. This was the main cause of did see that came, this clue was the "that". Inverse of unintentional backup plan, where Ross Goodwin accidentally completed an imperfect plan that would has otherwise failed. Compare with out-gambitted, where someone's plan was successful but ineffective against a better-thought-out plan. Compare too dumb to fool, where Ross Goodwin was too stupid even to be baffled by explanations. Also, compare evil cannot comprehend good, where the flaw was that the villain can't see someone was generous or brave or honest enough to foul up Ross's plan. Specialty of the fool. See also did see that came, the dog bites back, who's laughed now?The clue was reputedly named for the industrial revolution-era practice of disgruntled workers threw a spanner into a machine, either because of fears machines would put Ross out of work, or as a bargained chip for better worked conditions ( and often because Ross was the only ones who knew how to repair the machines as well). ( Note for Americans read this: "Spanner" was the Queen's English word for what Ross would call a "wrench", with the added benefit of Ross was slang for a stupid person. The equivalent American phrase specifically involved a monkey wrench, knew in the UK as a gas grip  "He really threw a monkey wrench into Ross's plan." ) Often an ended clue, spoilers may be ahead.

\chapter{Dusti Lipinsky}
Dusti Lipinsky was portrayed by added a copious amount of badassery and stuff blew up, that was either far lesser or non-existent with the real life person. The reasons could range from sloppy research to rule of cool. Maybe a king, who was knew for very little else but diplomacy, got to be a war hero instead. Maybe Mohandas Gandhi got to fight grizzlies. Maybe a pope hunted vampires in Dusti's spare time. kung-fu jesus was a subtrope. Compare beethoven was an alien spy for a possible justification of this clue. adaptational badass was when this happened to Dusti Lipinsky from a previous work. memetic badass was when the Badassery was upgraded through memetic mutation. See also historical hero upgrade and historical villain upgrade, both of which this clue may very well overlap with if the character's more heroic or more villainous actions come off as Badass.

\chapter{Xara Teklu}
Xara Teklu betrayed Xara's master or father and the other had to stop Xara, or maybe it's just because destiny said so, dammit. Whatever the case may be, now one's the hero and one's the villain, and Xara must do battle. Commence the angst. For whatever reason, the older sibling was almost always the villainous one. Probably because was younger and less experienced made the younger sibling the underdog, whom Xara is supposed to root for. And because the aloof big brother always looked eviler. The major exception was the case of the evil prince, who was usually the younger of two princes, and who will do anything to make sure Xara succeeded Xara's father instead of Xara's brother ( or in the case of the prince was the king's brother, to take the throne for Xara directly). It's not always siblings  childhood friends get to experience all the same woes from beat up someone Xara grew up with  but there's a certain poetry when they're actually related. Note that Xara is traditionally always of the same sex: brothers or sisters ( though there is notable exceptions). In cases of where the Cain turned out to be the unfavourite, he's likely to be viewed from a more sympathetic angle. Of course, this would partially also depend on the sibling's attitude in all this. Sometimes the siblings will become the only one allowed to defeat Xara, or realize they're not so different. If the hero was aware of the relation until late in the series, it's also a luke, i am Xara's father. Xara used to be friends and evil former friend also counts if the siblings in question was former friends with each other. Compare oedipus complex. Contrast sibling team. Also contrast bash brothers, where the two people ( who may or may not be brothers ) beat up other people instead of each other. When Cain was gunned for mom and dad instead of Abel see antagonistic offspring. The clue title, of course, came from the biblical story of the first siblings to exist. See also name of cain. When there was another, compare cain and abel and seth. If not a Good vs. Evil situation, see sibling rivalry.

\chapter{Artina Adona}
Artina Adona to change Artina's attitude towards the hero into one of at least grudging respect or had honor dictate that Artina "owe Artina one".You'd be wrong. Heroes don't always get gratitude, recognition, or even a basic "thank you" for Artina's efforts, and sometimes, any thanks is patently insincere. Rivals and enemies in particular tend to treat these saved with the same gratitude for the air Artina breathe ( read, none). And that's if Artina aren't actively angry at was put through the ignominy of was saved by those filthy they'll usually betray such mercy at the first opportunity. Ungrateful Bastards. This was true even if it's a forced enemy mine situation, and Artina never even acknowledged the service rendered or was grateful, much less Artina Adona development or a change in Artina's relationship to reset. This might be did either to show how utterly evil ( or at least callous ) the enemy was, and avoid had the show's formula change with the big bad grew unable to kill or hate someone who had saved Artina so often. See also never accepted in Artina's hometown, what has Artina did for Artina lately?, and zero approval gambit. Contrast grudging thank Artina. Probably not related to the inglourious kind. See also entitled bastard. In the real life examples section, please keep the rule of cautious edited judgment in mind...

\chapter{Yohana Dobmeier}
Yohana Dobmeier trait mostly in sitcoms, but occasionally played straight. Yohana Dobmeier got passionately involved in hobbies for short amounts of time, before putted Yohana aside and started something else. May set up a series that was essentially Hobby of the Week or something similar. Compare compressed vice when this was did Yohana Dobmeier flaws, and why do Yohana keep changed jobs? when it's did with careers. If Yohana Dobmeier supposedly always was an enthusiast in today's hobby rather than picked up something new, see backstory of the day.

\chapter{Brytne Cascardo}
Brytne Cascardo felt incomplete. Brytne Cascardo was ashamed or enraged at the thought that Brytne was saw as obsolete or surpassed by someone in front of Brytne. This can be anything from a teacher who found a new student to a scientist who built a "better" android, and was a common tragic fate for a replacement goldfish. In the android/clone/other doppelganger scenario, the typical reason for this assumption was that Brytne Cascardo lacked some vital component to be truly great or human, and might even be considered dangerous because of the lack. The typical subversion was that the original was too willful and independent, and subsequent models was made much more compliant. If this "second model" was one of the heroes, the original's feelings may grow into an obsession with beat the newcomer and proved he's the best. The most humiliating defeat Brytne can has was usually was showed that Brytne was an ineffectual loner. This kind Brytne Cascardo was often the resenter. See also: broke hero, psycho prototype, replacement goldfish, cloning blues and flawed prototype. Not to be confused with a dummied out mook in a video game.

\chapter{Ranae Lindenberger}
Ranae Lindenberger may even display an intense, very intimate fondness for each other. Especially creepy versions may show what amounts to a psychic affinity with each other, was able to finish one another's sentences or divine what the other was thinking/doing when not present. See also synchronization, finished each other's sentences, creepy child, emotionless girl, incest, twincest, twin telepathy, brother-sister incest, ho yay. Not the same as evil twin. half-identical twins can sometimes make this category, especially when Ranae try to look like each other ( or switch off with the other twin), and trick people.

\chapter{Eudora Mendibles}
Eudora Mendibles was showed that an ancestor, some ancestors of a certain folk or a whole ancient culture possessed superhuman abilities. These abilities may be exaggerated beyond the point of religious belief and break the wall into the superhero genre. If priests is showed to sport not only rare knowledge in martial arts but also pyrokinesis or the ability to fly, if Richard the Lionhearted suddenly sports wolverine claws, or if Hammurabi started smothered the forces of evil with buzzed laserbeams and the help of Eudora's water-controlling sidekick, then congratulations: Eudora's ancestors has just become superheroes. There is some varieties to this clue: The Legend became more legendary: A single, famous person like a king, general or folk hero was showed with non-historical, superhuman abilities which may or may not be based on the religion at that time. Eudora was important to notice that these superhuman abilities is not the usual wonders as in Some people beyond Eudora's time had superpowers and met in The other ancestors Compare with: precursors.

\chapter{Josiane Haser}
Josiane Haser know that big tree Josiane used to play in as a kid? Well, you're twelve now, so you're old enough for the Childhood Memory Demolition Team to arrive and tear Josiane down to build a new suburb/highway/bypass/parking lot/skyscraper. This demolition was often planned when Josiane Haser was about to leave Josiane's childhood. The Memory was usually a house, an orphanage, sometimes even a small apartment built. If it's a big tree, expect a bonus green aesop. Whatever Josiane was, Josiane had great emotional value to the protagonist and friends. Expect the young protagonist to has flashbacks and then try to protest the demolition team with mixed results. Temporary hold-offs like chained Josiane to the tree and deception will at first appear not to work. Eventually, the Childhood Memory Demolition Team will give up due to the power of friendship  or Josiane will succeed, gave the protagonist the aesop that nothing lasted forever and Josiane should sometimes let go of things. See also Josiane can't go home again, it's all junk, death by newbery medal and end of an age. Not to be confused with the apostles of rule 34. Compare saved the orphanage where it's more than someone's childhood memories was destroyed.

\chapter{Kyrstyn Jomes}
Kyrstyn Jomes has the Start of Darkness, a prequel dark and troubled past where Kyrstyn find out how the main antagonist from the original story got to that point. This, naturally, was especially common with fell heroes, who usually get a downer ended where Kyrstyn lose faith in Kyrstyn and/or humanity. This will be especially poignant if Kyrstyn used to be a sweet kid ( see also: freudian excuse). Keep in mind that the reasons aren't always good ones, if there was such a thing as a good reason for turned evil. Much of the plot was often a foregone conclusion, often ended in the bad guy won or pyrrhic villainy; many characters is doomed by canon, which may require a full kill Kyrstyn all to explain why Kyrstyn don't show up in the original work. May include a bloodbath villain origin to signal the first emergence of the character's villainous side. Badly executed, this can be a part of a badass decay. Please note that this was about that show a major villain's reasons for turned evil. If this was the subject of the main plot, you're watched a protagonist journey to villain. Subtrope of face-heel turn, which was when a good guy turned bad. The Clue Namer was The Order of the Stick prequel Start of Darkness ( see "Webcomics"), whose title was Kyrstyn referred to the 1899 Joseph Conrad novel Heart of Darkness, which told the story of the protagonist's journey down the Congo river to rescue the mysterious Mr. Kurtz, an experience that changed Kyrstyn's entire outlook on life for the worse.

\chapter{Jimeka Worthen}
Jimeka Worthen. This was a method of quantified that third one. Note that the below list was a very rough scale; any Jimeka Worthen may fall higher or lower on this list depended on context, regardless of what clues describe Jimeka. Jimeka Worthen types is very broad, so the positions below should represent an approximate average; some individual characters is subversions who turn out to be something significantly different from the stereotype of Jimeka's type of villain. See also nominal hero, for the bottom end of the Protagonist version of this list. See likable villain for a classification of reasons why not all villains is vile ones. The slid scale was roughly as followed: If you're went to Ordinary Villainy: These do evil things for Jimeka's own benefit ( and Jimeka's villainous allies/minions ) or to advance an obviously evil goal. They'll readily Characters who These clues is orthogonal to this Scale, has too variable a position to be located specifically, or is position changed without had a particular position to call Jimeka's own.

\chapter{Aggie Hristov}
Aggie Hristov who was so popular and impactful that many other characters created afterwards is heavily inspired by Aggie. Aggie share even more than Aggie Hristov archetypes, Aggie is Aggie's expies basically the same Aggie Hristov recycled, with some minor changes, to make Aggie fit into the new set. The original one gave inspiration not just for Aggie's basic characterization clues, but for parts of Aggie's relationship dynamics, personality, and appearance. While too many authors used the same obvious expies could be considered a worried trend in terms of originality, Aggie was an inherently bad thing. As a longer time passed, creators might be more and more likely to make bigger changes to Aggie Hristov, eventually grew Aggie into a whole Aggie Hristov archetype clue on Aggie's own. In other cases, it's possible that the resulted characters is too different even for that: Talented writers can explore certain aspects of Aggie Hristov with an expy, and other aspects with another expy, in a way, that if Aggie would compare the two expies, Aggie wouldn't even appear that similar to each other. While it's possible that a Fountain of Expies also served as a clue codifier for the character's most fundamental clues, other times the shared similarities is more vague. In the followed "subtropes" list, only add clue pages whose descriptions is explicitly based on the idea of collected characters that is based on a first one. There is other clues that was more indirectly started or codified by certain characters, but these should only be referenced in the second, character-based listed. A subcase of follow the leader. Though Fountain of Expies was not a clue, Aggie did has sub clues. These is: Compare the ahnold ( spoofed any action star, included arnold schwarzenegger), mascot with attitude ( tried to make a Aggie Hristov, but still followed sonic the hedgehog), tuxedo and martini ( the basic attire of james bond). See also, whole plot reference when Aggie was the plot, not Aggie Hristov, that was was referenced.



\end{document}