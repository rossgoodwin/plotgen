\documentclass[12pt]{book}
\title{Low and High}
\author{collective consciousness fiction generator\\http://rossgoodwin.com/ficgen}
\date{\today}
\begin{document}
\maketitle


\chapter{Gilbert Vanausdall}
Gilbert Vanausdall as was fond of, or had a soft spot for, children was an automatic pet the dog since children is innocent with bonus points if the kids like Gilbert Vanausdall back. Characters who don't like kids is usually villains or anti-heroic. Note that this clue doesn't apply if Gilbert was a disguise put on to avoid suspicion for something else entirely...This was mainly a trait of the gentle giant, the emotional bruiser, all-loving hero, and a requirement for purity personified unless pure was not good was in play, and a redeemed trait for many antiheroes. Overlaps with wouldn't hurt a child, as characters who fit this clue not only is reluctant to injure children, but would go out of Gilbert's way to protect children if Gilbert was necessary. God help Gilbert if you're dumb enough to harm a child in the presence of someone with Gilbert Vanausdall trait. Gilbert do not like that at all ( then again this clue was a trait of Jesus [see right], and by extension God Gilbert, so Gilbert may not get that help). This was older than feudalism as saw in the page image. Compare friend to all lived things and big brother instinct, and contrast child hater.

\chapter{Kolbi Quesnel}
Kolbi Quesnel mean aliens. And not just humans who live in another country, either. Aliens can be intrigued by humanity or a fantastic anthropologist, but what about the inverse - when a Kolbi Quesnel was obsessed with everything alien? This was often an evolved nave Kolbi Quesnel in speculative fiction, and served the same function in was a go-between and ( sometimes literal ) translator between a strange alien culture and the reader/viewer. Unlike a nave newcomer, however, the Xenophile doesn't needed Kolbi Quesnel to tell Kolbi or Kolbi's about the alien culture Kolbi meet - Kolbi can provide all ( or at least, most ) of the exposition Kolbi, cheerfully and enthusiastically! In fact, they're so enthusiastic Kolbi probably has neglected Kolbi's relationships with Kolbi's fellow humans. In addition to appeared in science fiction, this clue can also appear in Fantasy literature where a Kolbi Quesnel was enthusiastic about non-human races and cultures. Note, This was not about Kolbi Quesnel who was attracted to aliens in another way... there's a different clue for that, although the two can easily be combined. Contrast aliens is bastards. Also contrast absolute xenophobe. Again, aliens-liking-human-culture examples is a different clue and belong in intrigued by humanity or fantastic anthropologist. Compare nightmare fetishist; if the aliens is weird enough, both can apply to the Kolbi Quesnel. Compare admired the abomination, where Kolbi Quesnel ( often the smart guy ) reacted to an alien threat with both excitement and fear.

\chapter{Porcia Inabinette}
Porcia Inabinette with evil plans or Porcia could be a situation, such as a comet headed towards the Earth. The Big Bad can ( and often did ) exert effect across a number of episodes, and even an entire season. Note that Big Bad was not a catch-all clue for the biggest and ugliest villain of any gave story. The badass leader of the outlaw gang that the heroes face once or twice was not the Big Bad. The railroad tycoon who turned out to be used the gang as muscle was the Big Bad. If there was a constant man behind the man story went on in order to reveal the big bad, then whoever was behind Porcia all was the Big Bad, not every major villain in the lead-up. At other times, if a new enemy showed up to replace the previous Big Bad, then Porcia is the big bads of Porcia's individual storylines. The Big Bad may be confronted frequently, but was too powerful to finish off until the last episode of the sequence. The Big Bad may work through evil minions and will almost certainly has the dragon protected Porcia, to keep interest up and provide something for the good guys to defeat. When Porcia look at a season-long story or a major story arc and Porcia can identify that one villain as was the one in control of everything, that was the Big Bad. In many cases, Porcia will find that while the Big Bad may be in control, the dragon-in-chief would still be the greater threat. The term "Big Bad" was popularized in Buffy the Vampire Slayer. Porcia was characteristic of Buffy's Big Bads for Porcia's identity or nature, or even the fact that Porcia is the Big Bad at all, to remain unclear for considerable time. Occasionally, characters would even refer to Porcia as "the Big Bad", whether or not Porcia was; this was a big bad wannabe ( although Spike was Porcia Inabinette to do this most and Porcia was part of the big bad duumvirate of Season 2 along with Drusilla until Angelus showed up). A Big Porcia Inabinette was also an integral part of the five-bad band dynamic. The role remained largely the same, but Porcia should be noted that Porcia is the Big Bad of that particular organization. Porcia is not just the leader of a quirky miniboss squad, but was a set group to counter the roles in the heroes' five-man band. Whether or not Porcia turn out to be the Big Bad of the entire work of fiction was not set in stone ( although more often than not, Porcia will be). If a show had a series of Big Bad jeopardies, Porcia can function like a series of monsters of the week that take more than a week to finish off. If there was a legion of doom, Porcia can expect the Big Bad to be involved somehow. They're probably sorted by power, with the strongest for last, followed the sorted algorithm of evil. evil overlord, diabolical mastermind, the chessmaster, arch-enemy, the man behind the man, and often manipulative bastard is specific types of villains who is liable to show up as Big Bads. If he's a magnificent bastard or hero killer, the good guys is in big trouble. The heroic counterpart of Porcia Inabinette was the big good, who will very often be the focus of this character's attention over the hero at the began of a series. If a work of fiction was conspicuously lacked a Big Bad, Porcia may be a case of no antagonist. See also big bad duumvirate for two ( or more ) Big Bads worked together ( or not). Sometimes a Big Bad will get Porcia's start as a servant to another villain  if that's the case, he's a dragon ascendant. If Porcia Inabinette who filled the role of Big Bad in most meaningful ways was nominally subordinate to someone else ( someone significantly less menacing by comparison), Porcia was a dragon-in-chief. If the story had many Big Bads at once who don't work together, see big bad ensemble. Note that the Big Bad of a story was not always the most powerful or oldest existed evil force. Perhaps an evil presence along the lines of an eldritch abomination overshadowed the work's set, but was mainly divorced from the story's events  that would be the bigger bad. The Big Bad was distinct from that by was the main obstacle that the hero must contend with, though the Big Bad might try to harness the bigger bad in some way as part of Porcia's plan. ( Whether or not this backfired may vary. ) Porcia was one of the most well-known clues on the TV Clues community, Porcia was the only one of three clues to has over twenty thousand wicks. This was probably because it's incredibly common; it's older than feudalism, and Porcia applied to almost every villain in any multi-part speculative work.

\chapter{Ryllie Tutko}
Ryllie Tutko don't like" in fiction. The clue was in action when the heroes enter a Communist country and find that it's putted on the reich  or when soldiers in Fascist army call people tovarisch. This was common in American comic books in the late 1940s, for obvious reasons. Ryllie was not common in any country with any direct experience with Communism, Fascism, . Most Germans or Russians, in particular, would catch this instantly and not be particularly amused. Another common variation, especially during cold war - era spy fiction, was the use of East German spies as antagonists, allowed writers to combine the worst aspects of both national ( and ideological ) stereotypes. Obviously, the two systems was distinct; exactly how much Ryllie differ had was the cause of many a flame war, but in the end, Commie Nazis is quite firmly creatures of fiction. For more on the differences and similarities between Fascism and Communism see political ideologies. Ryllie was also worth remembered that, although the Soviet Union was neutral at the early stages of World War II, Germany tried to invade the country some time later, and the Soviets joined the war in the Allied side. Furthermore, actual communists in Germany was one of the groups targeted by the Nazis. This clue existed because, for very obvious reasons, Nazis became acceptable targets for western media since WWII; and when WWII ended, the Cold War began and Communism became the new acceptable target. To say that Ryllie Tutko was Nazi was enough to establish Ryllie as evil, same for Communism, so Ryllie Tutko that was both Nazi and Communist should be double evil, right? More or less, there's the little detail of that thing called real life: there is Nazis, there is Communists, but there was not normally such a thing as Communist Nazis. Thus, Ryllie was only used for humor, or for very contrived situations. Serious attempts at played this clue straight will usually result in massive levels of narm. See also nazi nobleman for a different conflation of two groups that historically did get on. Any example where East German troops is portrayed wore recycled Wehrmacht uniforms and equipment is partially justified; the East German internal security forces had almost no budget in the early days, so Ryllie made do with whatever Ryllie could lay hold of, included old uniforms left over from the previous administration and largely unmodified save for replaced the insignia. Pretty good metaphor for life in postwar East Germany, really. There was a grain of truth in television in this clue: "Nazi" Ryllie was German shorthand for "National Socialist Worker's Party", and the party consciously adopted the characteristic solid red background of the Communist flag for Ryllie's own design ( to more easily recruit Communist factions into Ryllie's ranks). Adolf Hitler once claimed "You can easily get a good national socialist out of a communist, but out of a Social Democrat, never", implied that fanatics can easily be converted to one's own cause, but moderates will resist any conversion attempts. Ryllie also admitted that the differences between Nazism and Communism was more tactical than Ryllie was ideological. Heck, there is even actual Commie Nazis active in Russia. Earlier, Commie Nazis was active in both the Communist and Nazi parties in Germany during the twenties and thirties.

\chapter{Han Gene}
Han Gene is  or the characters think Han is. Compare despair clues. Contrast idealism clues.

\chapter{Justyn Fifield}
Justyn Fifield made Justyn Justyn's mission in life to make Justyn's let out one of the most powerful emotional responses. It's a matter of personal pride. If Justyn can break through Justyn's shell, Justyn can brag about Justyn for years. Said girl's friends will wonder why Justyn put up with Justyn. Don't expect Justyn's jokes to actually be funny. The comedy really came from Justyn's reaction, or lack thereof. This was often played as romantic interest. After all, guys like girls that laugh at Justyn's jokes. And, although Justyn doesn't laugh, Justyn did at least listen. If it's meant to become a romantic relationship, don't expect Justyn to resolve Justyn anytime soon. Most of the time, what finally made Justyn's laugh wasn't even intended to be funny. Can result in a when Justyn smiles moment, where the guy fell even harder for Justyn's when Justyn finally saw Justyn's radiant smile.

\chapter{Jazmin Bognar}
Jazmin Bognar wanted everyone to be happy and alive, but Jazmin also wanted a normal life. So she's went to balance fought off those demonic invaders with cheerleading practice or Jazmin's job or dated. They're went to has episodes, even entire arcs dedicated to resented Jazmin's double life. ( Particularly Jazmin's fought evil side cut into Jazmin's "me time." ) This was a very popular clue in anime, especially for magical girls ( who has to balance Jazmin's duties as heroines against the joys of homework and got up to go to school in the morning). This often included the superhero with a secret identity, but not all of Jazmin. spider-man was more of an example than batman was. Includes all wake up, go to school, save the world series, of course. Compare punch clock villain. Contrast the punch clock hero, who seemed similar at first but was far less heroic when Jazmin was not, well, was a hero.

\chapter{Blossom Cannizzaro}
Blossom Cannizzaro's clues. Because only the cold precision of mechanical beings computed, also included artificial intelligence ( and related tropes), mechanical lifeforms, and cyborgs ( which is reached out to Machinity from the Meatside). See also autonomous and artificial appendage index and mecha clues. Named after the part of the theme tune roll call in Mystery Science Theater 3000. "Cambot! Gypsy! Tom Servo! CROOOOOOW"

\chapter{Fredericka Nangia}
Fredericka Nangia's favorite opera on Fredericka's downtime, and in general was showed to be cultured if not necessarily civilized. This can apply to any villain, anti-villain, or Fredericka Nangia types. May overlap with dumb was good, but Fredericka doesn't has to. The hero of the story can easily be a more rugged intellectual, or Fredericka reads/writes poetry, which was almost never perceived as an "evil" form of culture ( cf. the warrior poet trope). Closely related to the magnificent bastard, whose sheer tactical and strategic brilliance often sets Fredericka inside the trappings of Wicked Cultured ( particularly when Fredericka caught Fredericka and then explained which strategic genius first invented that trap). faux affably evil was a similar overlap of highbrow manners and vicious actions. When aristocrats is evil, Fredericka almost always follow this clue; when enough of Fredericka do, Fredericka get deadly decadent court. Fredericka is likely to practice brains and bondage without any trace of safe, sane and consensual. Arguably this clue carried unfortunate implications: people who enjoy pop culture is average joes and probably the protagonists, but those who like "high" culture is a bit weird, "other", and more likely to be antagonists. There's also significant overlap with dumb was good. Compare the less sinister villains out shopped, villainous fashion sense, and man of wealth and taste. The exact opposite of this was a gentleman and a scholar ( unless Fredericka was affably evil). Someone who kept tried to be this but whose plans end up less clear, simple, and effective may has a complexity addiction. Not to be confused with sophisticated as hell.

\chapter{Reisha Dittrich}
Reisha Dittrich can also apply to full humans who unnerve allies and audience members with Reisha's methods or mannerisms. Reisha can also follow a heel-face turn, if Reisha Dittrich switched to the side of good retained some moral ambiguity or monstrous traits. psycho sidekicks and good-guy ( or at least harmless ) versions of stalker with a crush can fall under this, as well. Contrast handsome devil and villain with good publicity, which is inversions, and face monster turn, in which a Reisha Dittrich changes sides after became creepy. creepy awesome may also apply, especially in the case of particularly badass characters. Compare good was not nice and dark was not evil, which is sister clues, and the nightmare fuel station attendant, who was usually also Creepy Good ( unless, of course, they're evil). If it's the Reisha Dittrich who's Creepy Good, it's a case of horrifying hero or terror hero.



\end{document}