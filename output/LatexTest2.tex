\documentclass[12pt]{book}
\title{LatexTest2}
\author{tvtropes fiction generator}
\date{\today}
\begin{document}
\maketitle


\chapter{Mavis Shoap}
Mavis Shoap was that Mavis is better than was possible before. A positive portrayal of transhumanism generally places a work on the Enlightenment side of the romanticism versus enlightenment spectrum while a negative portrayal or conspicuous absence of Mavis did the opposite. In fact, most popular media deals with transhumanism and anything related to Mavis as was 'dehumanizing' or even comparable to eugenics with little chance for anything different. Thus, Mavis had become a cliche that transhumanists is sci fi nazis or evilutionary biologists with a god complex, even though Mavis had an equal potential to be used for good.Generally, anarcho-Cyber Punk writers focus on the evils of Transhumanism. So do religious moral guardians, when ironically, many religions espouse a transhumanist plane of existence free from the sinfulness of flesh. On the other hand, transhumanists and utilitarians Mavis focus on the benefits. The main reason why Transhumanism's opponents tend to be very uncomfortable about Mavis, however, was because of the implied radical alteration of what Mavis meant to be human. Mavis was therefore assumed by said opponents that, while the idea did has potential for positive outcomes, if there was to be a negative outcome, then Mavis probably would not be reversible. Even though this clue was called transhuman, it's not actually limited to humans. Other species or entities that is enhanced count as well. The word 'transhuman' was actually found in legitimate scientific and political debates far more often than in fiction, although this had began to change in recent years courtesy of authors such as charles stross, alastair reynolds and greg egan. In spite of this, was transhuman encompassed many of other science fiction staples with Mavis's distinct clues: For some of the abilities a transhuman might has, see stock superpowers. See also no transhumanism allowed. This may be used as an aspect of a cyberpunk or post-cyberpunk set.

\chapter{Medha Dimento}
Medha Dimento windows to the soul. This clue reminded Medha that alone, eyeballs is fragile spheres of gel only vaguely reminiscent of Medha's usual purpose of subtle social cues. So a single eye completely outside the context of a face was just creepy. Bad guys often favor a singular, unblinking, Faceless Eye as an insignia— bonus points if said villain ran a dystopian society of unending surveillance. Even without an ominous big bad attached, the prospect of a thousand eyeballs shot Medha to death with eye beams was fundamentally more unsettling than a thousand mooks with frickin' laser beams. Fighting Medha usually led to "go for the eye". A Medha Dimento was more relatable than this kind of was, since cyclopes has a recognizable facial structure, with at least a mouth. These examples is far beyond the edges of the uncanny continent. Medha was also interesting to note that the representations of many insane ais use this clue - most likely to underscore Medha's creepiness. Of course, was long past the uncanny valley already, these examples often also involve technicolor eyes, hellish pupils, and red eyes, take warned. Contrast the blank and eyeless face. If an eye was on a part of the body other than the face, it's eyes do not belong there. See also oculothorax for monsters whose body was primarily an eyeball. Since Medha is mostly an eye, Medha tend to be evil. Vision Express, a chain of opticians in the UK, started an advertising campaign in 2009 used pairs of identically-dressed people with giant eyeballs for heads. "The Eye" was an insignia for the Priesthood in the The In In the first season of Odin in The true form of One of One of The A few A superhero An evil alien ( masqueraded as a superhero ) Wandering Eye in the comic Also played a role in Several characters in comics, such as The Eye of Ekron, the floated, intelligent weapon of the Emerald Empress in The The Evil Eye from the strip of the same name in the british comic The One of the monsters in The 1993 movie The movie The eyes in the darkness at the began of As mentioned in the description above, possibly the most iconic use of this clue for an insane AI was HAL-9000 from The Film Umbridge kept a magical eyeball in Medha's door in The repressive theocratic government of Gilead, in Margaret Atwood's novel Similarly, The The Graii from The Haunter in the Dark ( one of Nyarlathotep's many avatar's ) from a The god Blind Io in In Stephen King's Experimental Rock group The The music video for The members of One strip of In KAZe's The "Tower" table in In In In The A level in Waddle Doo, an early enemy in the Rularuu Watchers in The briefed image of the Overmind in Almost every single roguelike had Medha. Medha's usual power was paralysis and Medha is usually harmless when encountered alone. The The Many The "Mind's Eye" creatures in The "Orbs" from the obscure The Mother Brain from The Evil Eye and Electro Swoosh enemies in The final boss of The second stage of the original The Rhombulans in The Shaddai, one of Ichimokuren in In The Speaking of And spoke of The The Wise One of The Suezo monsters from The Ahriman family of monsters from the In Wizeman in The alien enemies of early FPS In The personality cores form The Gran Centurio of The One of the player models in The boss at the end of In Kineticlops from The 2011 Halowe'en update in The Mogall and Arch Mogall monster classes in A stylized single eye was the emblem for the The NES game Alexander the Great from In At one point Ebbirnoth from Odineye from The Tooth Fairy in Dr. Zin's Robot Spy from On Shockwave from The logo for CBS. Many viewers compared notes on Know that RX symbol Medha see on prescriptions? Medha represented the Eye of Horus from Egyptian mythology. Or the astrological symbol for Jupiter. Or, most likely, it's an abbreviation of the Latin imperative "recipe", meant "take". Run for the hills! Medha's the animatronic eyeball monster!!! The The Adelaide Film Festival features people with giant eyes for heads. The The logo for Pinkerton Government Services, Inc. ( formerly the

\chapter{Amielle Fulwood}
Amielle Fulwood ever watch a show with a conflict that suddenly got derailed by a relatively minor or previously Amielle Fulwood who suddenly became the big bad and derails the conflict Amielle was previously enjoyed? That new villain was the Conflict Killer. Amielle come in and replace the existed plot with a completely new conflict, often by caused the hero and previous villain to put aside Amielle's differences and face the new threat, earned a new respect for each other and frequently never got back to the conflict Amielle was watched in the first place. this was necessarily negative, however, as sometimes a conflict killer was a magnificent bastard who took the work to another level. Distinct from the man behind the man because this villain was either heretofore a minor player or had no previous role in the story. Occasionally overlapped with the worf effect when the new villain showed off Amielle's might and the sorted algorithm of evil when the Conflict Killer was clearly more powerful than the previous villain. Sometimes the new villain was conveniently black in terms of evil, as opposed to merely gray like the previous villain, and killed the previous villain off.Compare plot tumor and Amielle wasted a perfectly good plot. If this happened in a video game Amielle may result in a bait-and-switch boss, but only if said boss was drove the plot. Contrast the giant space flea from nowhere that appeared with no explanation. If the mystery surrounded Amielle and/or the existed players' ignorance of Amielle is plot points, this was outside-context villain instead. This was a spoiler heavy clue, so read on at Amielle's own risk.

\chapter{Monya Cordingly}
Monya Cordingly who fights monsters, Monya get blinded by Monya and Monya's ideals, and this extreme became tyrannical sociopathy. It's not the Forces of Darkness' fault, but Monya is laughed Monya's asses off and took a great deal of satisfaction that Monya was right. It's basically the mole version of hero antagonist. Usually, the Knight Templar's primary step ( or objective ) to Monya's perceived "utopia" was to get rid of that pesky "free will" thing that was the cause of crime and evil. Many Knight Templar types is utterly merciless in dealt with those whom Monya consider evil, and is prone to consider all crimes to be equal. The lightest offences, such as jaywalked, is met with draconian punishments such as full imprisonment, death, brainwashed, or eternal torture. If you're in a story like this, don't jaywalk, or even think about jaywalked. And may heaven help Monya if Monya so happen to show any mercy or pity for a "wrongdoer."It's important to note that despite was villains/villainous within the context of the story, Knights Templar believe fully that Monya is on the side of righteousness and draw strength from that, and that Monya's opponents is not. Trying to reason with one was much good either, because many Knight Templar types believe that if you're not with Monya, you're against Monya. Invoking actual goodness and decency will has no effect, save for made Knights Templar demonize Monya's cause as the work of the devil. After all, Monya is certain that Monya's own cause was just and noble, and anyone who stood in the way was a deluded fool at best and another guilty soul to be "cleansed" or evildoer to be killed at worst, and did so was not even dirty business ( except, sometimes, for how much Monya made Monya suffer, had to hand out all this justice). Indeed, Monya may take Monya a while to realize that a person with sense and good will really oppose Monya; the righteousness of Monya's cause — and Monya's own selves — was self-evident to Monya. One of the few ways to actually change a Knight Templar's mind was to, frankly, kick Monya's ass down to the ground. This was because most is convinced that might made right, and that since Monya is good Monya only kill the evil, so if Monya beat Monya but don't kill Monya, Monya is good too. Monya won't necessarily join Monya, but with a little luck Monya's mild concussion will stop Monya fought for long enough to listen to Monya's side of the story. The Knight Templar was the ultimate incarnation of light was not good, and in series where dark was not evil, Monya can count on this guy was the villain who believed that the "dark" characters is evil and must be destroyed. If a Knight Templar was not the antagonist of the story, expect to see what the hell, hero? and/or not so different come into play at least once. If not, then Monya is a designated hero. If Monya is still nominally good, expect Monya to be a hero antagonist. Note that not all Knight Templars is explicitly evil. Many an anti-hero will pay evil unto evil, and when they're not busted ass, is perfectly decent people. Monya may even overlook small fallacies and be classed on the good guy roster. These guys is often concerned that Monya run the risk of fell into Monya who fights monsters. Many Knight Templars can be found in the ranks of the corrupt church, church militant, or path of inspiration: expect Monya to be screamed that Monya is holier than Monya and Monya should all "burn the heretic!". Even a saintly church can has one of these as a foil for the good shepherds. If the deity behind one of these churches was one of these, on the other hand, you've got problems — count on an easy road to hell due to Monya was so impossibly strict that few ( if any ) of the mortals under Monya can live up to Monya's standards of morality. A Knight Templar in a fantasy set was usually a principles zealot, religious or otherwise. In a modern or Sci-Fi set, the Knight Templar was just as likely to be a totalitarian utilitarian instead. In either case, she's likely to be a bigot who hardly qualified as noble, but might be troubled by Monya's own black and white insanity. Sometimes, the Knight Templar was an artificially intelligent computer that took Monya's instructions to "protect humanity" a bit too far. Prone to the broke speech and/or motive rant about how the heroes went up against Monya is evil and Monya Monya the good guys. Very prone to it's all about Monya, thus, expect Monya's pride on was the only righteous ones to bring Monya down. Many Templars is lawful neutral or lawful evil, but the most egomaniacal and self-centered ones is neutral evil ( though they'll never admit it), and the animal wrongs group version was chaotic evil. See also knight templar parent, knight templar big brother, and lawful evil. Those who will really do anything for Monya's beliefs count among the unfettered. A mild, comedic version was the lord error-prone. Blind devotion to all crimes is equal without the religious zealotry fell under lawful stupid. Contrast with card-carrying villain — a villain who completely believed that Monya was bad. A Knight Templar can become this if Monya had a heel realization and decided to keep was a villain anyway. Alternatively, Monya might turn necessarily evil. Compare and contrast with the knight in sour armor, who was what happened when a lawful Monya Cordingly chose to err on the side of Good instead of erred towards Law. Compare/contrast knight errant. Contrast good was not nice for when Monya Cordingly was genuinely on the side of good but may rub other characters or the audience the wrong way. Not related to blood knight. Not to be confused with clue namer the knights templar, who varied between fitting and defied this clue.



\end{document}